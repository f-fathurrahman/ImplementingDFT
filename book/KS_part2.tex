\chapter{Kohn-Sham equation part II}
\label{chap:ks_part_2}

In the previous chapters we have used pseudopotentials to model the interaction
between electrons and nuclei. The pseudopotentials that we have used
are limited to local form. In this case, the potential operator is diagonal
in real space. It turns out that it is very difficult to construct local pseudopotentials
that have good accuracy and transferability. Most of the pseudopotentials that
are used in density functional calculations have nonlocal components.

In this chapter, we will extend our code to handle
nonlocal pseudopotential term.
We will specifically use a family of pseudopotentials that was
proposed by Goedecker-Teter-Hutter (GTH) \cite{Goedecker1996} in 1996.
We have used local-only component of this pseudopotential in the previous chapters, so
our study is limited to the elements for which this pseudopotential has only local
component. Now, we will start to consider the nonlocal component.
This will enable us to use all elements present in the GTH pseudopotential set.

\section{GTH pseudopotentials}

The GTH pseudopotentials are used to model the electron-nuclei potential operator.
They can be written in terms of
local $V^{\mathrm{PS}}_{\mathrm{loc}}$ and
angular momentum $l$ dependent
nonlocal components $\Delta V^{\mathrm{PS}}_{l}$:
\begin{equation}
V_{\mathrm{ele-nuc}}(\mathbf{r},\mathbf{r}') =
\sum_{I} \left[
V^{\mathrm{PS}}_{\mathrm{loc}}(\mathbf{r}-\mathbf{R}_{I}) +
\sum_{l=0}^{l_{\mathrm{max}}}
V^{\mathrm{PS}}_{l}(\mathbf{r}-\mathrm{R}_{I},\mathbf{r}'-\mathbf{R}_{I})
\right]
\end{equation}
%
The local pseudopotential for
$I$-th atom, $V^{\mathrm{PS}}_{\mathrm{loc}}(\mathbf{r}-\mathbf{R}_{I})$,
is radially symmetric
function with the following radial form
\begin{equation}
V^{\mathrm{PS}}_{\mathrm{loc}}(r) =
-\frac{Z_{\mathrm{val}}}{r}\mathrm{erf}\left[
\frac{\bar{r}}{\sqrt{2}} \right] +
\exp\left[-\frac{1}{2}\bar{r}^2\right]\left(
C_{1} + C_{2}\bar{r}^2 + C_{3}\bar{r}^4 + C_{4}\bar{r}^6
\right)
\label{eq:V_ps_loc_R}
\end{equation}
%
with $\bar{r}=r/r_{\mathrm{loc}}$ and $r_{\mathrm{loc}}$, $Z_{\mathrm{val}}$,
$C_{1}$, $C_{2}$, $C_{3}$ and $C_{4}$ are the corresponding pseudopotential
parameters.
%
In $\mathbf{G}$-space, the GTH local pseudopotential can be written as
\begin{multline}
V^{\mathrm{PS}}_{\mathrm{loc}}(G) = -\frac{4\pi}{\Omega}\frac{Z_{\mathrm{val}}}{G^2}
\exp\left[-\frac{x^2}{2}\right] +
\sqrt{8\pi^3} \frac{r^{3}_{\mathrm{loc}}}{\Omega}\exp\left[-\frac{x^2}{2}\right]\times\\
\left( C_{1} + C_{2}(3 - x^2) + C_{3}(15 - 10x^2 + x^4) + C_{4}(105 - 105x^2 + 21x^4 - x^6) \right)
\label{eq:V_ps_loc_G}
\end{multline}
where $x=G r_{\mathrm{loc}}$.
%
The nonlocal component of GTH pseudopotential can written in real space as
\begin{equation}
V^{\mathrm{PS}}_{l}(\mathbf{r}-\mathbf{R}_{I},\mathbf{r}'-\mathbf{R}_{I}) =
\sum_{\mu=1}^{N_{l}} \sum_{\nu=1}^{N_{l}} \sum_{m=-l}^{l}
\beta_{\mu lm}(\mathbf{r}-\mathbf{R}_{I})\,
h^{l}_{\mu\nu}\,
\beta^{*}_{\nu lm}(\mathbf{r}'-\mathbf{R}_{I})
\end{equation}
where $\beta_{\mu lm}(\mathbf{r})$ are atomic-centered projector functions
\begin{equation}
\beta_{\mu lm}(\mathbf{r}) = 
p^{l}_{\mu}(r) Y_{lm}(\hat{\mathbf{r}})
\label{eq:betaNL_R}
\end{equation}
%
and $h^{l}_{\mu\nu}$ are the pseudopotential parameters and
$Y_{lm}$ are the spherical harmonics. Number of projectors per angular
momentum $N_{l}$ may take value up to 3 projectors.
The projectors can be written as
\begin{equation}
p^{l}_{\mu}(r) = \frac{\sqrt{2}}
{r^{l+(4i-1)/2}_{l}\sqrt{\Gamma(l + (4i-1)/2)}} r^{l+2(i-1)}
\exp\left[-\dfrac{r^2}{2r^{2}_{l}}\right] \, ,
\label{eq:proj_NL_R}
\end{equation}
where $\Gamma(x)$ is the gamma function. The projectors are normalized according
to
\begin{equation}
\int_{0}^{\infty} r^2\,p^{l}_{i}(r)\,p^{l}_{i}(r)\,\mathrm{d}r = 1 \, .
\end{equation}

In the case of periodic sytem, the local part of the pseudopotential
is constructed using the formula in the $\mathbf{G}$-space 
and transformed them back to real space.
We refer the readers to the original
reference \cite{Goedecker1996} and the book \cite{Marx2009}
for more information about GTH pseudopotentials.

Due to the separation of local and non-local components of electrons-nuclei
interaction, interaction energy between electron and nuclei can be decomposed as
\begin{equation}
E_{\mathrm{ele-nuc}} = E^{\mathrm{PS}}_{\mathrm{loc}}
+ E^{\mathrm{PS}}_{\mathrm{nloc}}
\end{equation}
%
where the local pseudopotential contribution is
\begin{equation}
E^{\mathrm{PS}}_{\mathrm{loc}} =
\int_{\Omega} \rho(\mathbf{r})\,V^{\mathrm{PS}}_{\mathrm{loc}}(\mathbf{r})\,
\mathrm{d}\mathbf{r}
\end{equation}
%
and the nonlocal contribution is
\begin{equation}
E^{\mathrm{PS}}_{\mathrm{nloc}} = 
\sum_{i}
f_{i}
\int_{\Omega}\,
\psi^{*}_{i}(\mathbf{r})
\left[
\sum_{I}\sum_{l=0}^{l_{\mathrm{max}}}
V^{\mathrm{PS}}_{l}(\mathbf{r}-\mathbf{R}_{I},\mathbf{r}'-\mathbf{R}_{I})
\right]
\psi_{i}(\mathbf{r})
\,\mathrm{d}\mathbf{r} \,\mathrm{d}\mathbf{r}'.
\end{equation}

\section{Data structures for nonlocal pseudopotential}

\subsection{Storing pseudopotential parameter}

We will consider how to store various information about GTH pseudopotential.
First we consider the type \jlinline{PsPot_GTH}. This type stores GTH pseudopotential
parameters for one element.

\begin{juliacode}
struct PsPot_GTH
  pspfile::String
  atsymb::String
  zval::Int64
  rlocal::Float64
  rc::Array{Float64,1}  # indexed by l, originally [0:3]
  c::Array{Float64,1}   # coefficients in local pseudopotential
  h::Array{Float64,3}   # indexed l, 1:3,1:3
  lmax::Int64           # l = 0, 1, 2, 3 (s, p, d, f)
  Nproj_l::Array{Int64,1}  # originally 0:3
  rcut_NL::Array{Float64,1}  # originally 0:3, needed for real space evaluation
end
\end{juliacode}

Example constructor
\begin{juliacode}
pspot = PsPot_GTH("C-q4.gth")
\end{juliacode}

Several important functions for our case
\begin{itemize}
\item \jlinline{eval_V_loc_R} and \jlinline{eval_V_loc_G}: evaluates local component
  of GTH pseudopotential in real (Equation \eqref{eq:V_loc_GTH})
  and reciprocal space (Equation \eqref{eq:V_ps_loc_G}), respectively.
\item \jlinline{eval_proj_R}: evaluates nonlocal projectors
  of GTH pseudopotential in real (Equation \eqref{eq:proj_NL_R}).
  There is also also the function \jlinline{eval_proj_G} which evaluates the
  projector in reciprocal space, however it is usually used for plane wave basis.
\end{itemize}


\subsection{Nonlocal Hamiltonian}

The type \jlinline{PsPotNL} stores several important quantities for Hamiltonian
evaluation of nonlocal pseudopotential.
%
\begin{juliacode}
struct PsPotNL
  NbetaNL::Int64
  prj2beta::Array{Int64,4}
  betaNL::Array{Float64,2}
end
\end{juliacode}
%
The most important field of \jlinline{PsPotNL} is \jlinline{betaNL}. This array stores
the atomic centered projector functions defined in Equation \eqref{eq:betaNL_R}.
\jlinline{NbetaNL} is number of such functions which mush be calculated for
all atoms. The array \jlinline{prj2beta} defines index mapping between projector
functions for all atoms to their atomic-species-specific data. The constructor
for \jlinline{PsPotNL} will be listed below.

\begin{juliacode}
function PsPotNL( atoms::Atoms, pspots::Array{PsPot_GTH,1}, grid; check_norm=false )
  Natoms = atoms.Natoms
  atm2species = atoms.atm2species
  atpos = atoms.positions

  prj2beta = Array{Int64}(undef,3,Natoms,4,7)
  prj2beta[:] .= -1   # set to invalid index

  # Here we calculate NbetaNL and define fill up prj2beta array
  # for mapping between betaNL index.
  # In case of prj2beta, we probably can do this better by using
  # prj2beta[1:Natoms][iprj,lm indices]
  NbetaNL = 0
  for ia = 1:Natoms
    isp = atm2species[ia]
    psp = pspots[isp]
    for l = 0:psp.lmax
      for iprj = 1:psp.Nproj_l[l+1]
        for m = -l:l
          NbetaNL = NbetaNL + 1
          prj2beta[iprj,ia,l+1,m+psp.lmax+1] = NbetaNL
        end
      end
    end
  end

  # No nonlocal components
  if NbetaNL == 0
    # return dummy PsPotNL
    betaNL = zeros(Float64,1,1)
    return PsPotNL( 0, zeros(Int64,1,1,1,1), betaNL )
  end

  Npoints = grid.Npoints
  betaNL = zeros(Float64, Npoints, NbetaNL)
  setup_betaNL!( atoms, grid, pspots, betaNL )

  return PsPotNL( NbetaNL, prj2beta, betaNL )
end
\end{juliacode}

The function \jlinline{setup_betaNL!} is listed below.
\begin{juliacode}
function setup_betaNL!( atoms, grid, pspots, betaNL )
  # ... snipped, various shortcuts

  ibeta = 0
  dr = zeros(3)
  for ia = 1:Natoms
    isp = atm2species[ia]
    psp = pspots[isp]
    for l = 0:psp.lmax, m = -l:l, iprj = 1:psp.Nproj_l[l+1]
      ibeta = ibeta + 1
      for ip in 1:Npoints
        dr[1] = grid.r[1,ip] - atoms.positions[1,ia]
        dr[2] = grid.r[2,ip] - atoms.positions[2,ia]
        dr[3] = grid.r[3,ip] - atoms.positions[3,ia]
        drm = sqrt( dr[1]^2 + dr[2]^2 + dr[3]^2 )
        betaNL[ip,ibeta] = Ylm_real(l, m, dr)*eval_proj_R(psp, l, iprj, drm)
      end
    end
  end
  return
end
\end{juliacode}

For system without any nonlocal pseudopotentials we provide a dummy function:
\begin{juliacode}
function PsPotNL()
  betaNL = zeros(Float64,1,1)
  return PsPotNL( 0, zeros(Int64,1,1,1,1), betaNL )
end
\end{juliacode}

Using these information, we can
\begin{juliacode}
function calc_E_Ps_nloc( Ham::Hamiltonian, psi::Array{Float64,2} )
  
  # ... snipped, various shortcuts

  betaNL_psi = psi' * Ham.pspotNL.betaNL * dVol

  E_Ps_nloc = 0.0
  for ist = 1:Nstates
    enl1 = 0.0
    for ia = 1:Natoms
      isp = atm2species[ia]
      psp = pspots[isp]
      for l = 0:psp.lmax, m = -l:l
        for iprj = 1:psp.Nproj_l[l+1], jprj = 1:psp.Nproj_l[l+1]
          ibeta = prj2beta[iprj,ia,l+1,m+psp.lmax+1]
          jbeta = prj2beta[jprj,ia,l+1,m+psp.lmax+1]
          hij = psp.h[l+1,iprj,jprj]
          enl1 = enl1 + hij * betaNL_psi[ist,ibeta] * betaNL_psi[ist,jbeta]
        end # jprj, iprj
      end # m, l
    end
    E_Ps_nloc = E_Ps_nloc + Focc[ist]*enl1
  end
  return E_Ps_nloc
end
\end{juliacode}

Nonlocal operator:
\begin{juliacode}
function op_V_Ps_nloc( Ham::Hamiltonian, psi::Array{Float64,2} )
  # ... snipped

  betaNL_psi = psi' * Ham.pspotNL.betaNL * dVol
  
  Vpsi = zeros(Float64,Npoints,Nstates)
  for ia = 1:Natoms
    isp = atm2species[ia]
    psp = pspots[isp]
    for l = 0:psp.lmax, m = -l:l
      for iprj = 1:psp.Nproj_l[l+1]
        ibeta = prj2beta[iprj,ia,l+1,m+psp.lmax+1]
        for jprj = 1:psp.Nproj_l[l+1]
          jbeta = prj2beta[jprj,ia,l+1,m+psp.lmax+1]
          hij = psp.h[l+1,iprj,jprj]
          for ist in 1:Nstates
            cc = betaNL_psi[ist,jbeta]*hij
            # should only loop over limited points (for which the projectors are nonzero)
            for ip in 1:Npoints
              Vpsi[ip,ist] = Vpsi[ip,ist] + betaNL[ip,ibeta]*cc
            end
          end
        end # iprj
      end # jprj
    end # m, l
  end
  return Vpsi
end
\end{juliacode}

Hamiltonian operator:
\begin{juliacode}
function op_H( Ham::Hamiltonian, psi::Matrix{Float64} )
  Nbasis = size(psi,1)
  Nstates = size(psi,2)
  Hpsi = -0.5*Ham.Laplacian*psi
  if Ham.pspotNL.NbetaNL > 0
    Vnlpsi = op_V_Ps_nloc(Ham, psi)
    for ist in 1:Nstates, ip in 1:Nbasis
      Hpsi[ip,ist] = Hpsi[ip,ist] + ( Ham.V_Ps_loc[ip] +
        Ham.V_Hartree[ip] + Ham.V_XC[ip] ) * psi[ip,ist] + Vnlpsi[ip,ist]
    end
  else # no nonlocal pspot components
    for ist in 1:Nstates, ip in 1:Nbasis
      Hpsi[ip,ist] = Hpsi[ip,ist] + ( Ham.V_Ps_loc[ip] +
          Ham.V_Hartree[ip] + Ham.V_XC[ip] ) * psi[ip,ist]
    end
  end
  return Hpsi
end
\end{juliacode}



\section{Validation}

Comparison with Octopus calculations.

\subsection{LiH}

N = 40, spacing = 0.41026
\begin{textcode}
Emin_PCG step       25 =      -0.7670931285  6.9040618e-07
Emin_PCG step       26 =      -0.7670933986  2.7002749e-07

Emin_PCG_dot is converged in iter: 26

----------------------------
Final Kohn-Sham eigenvalues:
----------------------------

  1      -0.1609715031

----------------------------
Final Kohn-Sham energies:
----------------------------

Kinetic =       0.6405954267
Ps_loc  =      -2.2606439171
Ps_nloc =       0.0431343816
Hartree =       0.9469417869
XC      =      -0.4899058838
NN      =       0.3527848071
----------------------------
Total   =      -0.7670933986
\end{textcode}

Octopus result
\begin{textcode}
*********************** SCF CYCLE ITER #   15 ************************
 etot  = -7.86985069E-01 abs_ev   =  8.84E-07 rel_ev   =  2.74E-06
 ediff =       -3.96E-11 abs_dens =  1.54E-05 rel_dens =  7.71E-06
Matrix vector products:      3
Converged eigenvectors:      1

#  State  Eigenvalue [H]  Occupation    Error
      1       -0.161349    2.000000   (7.4E-07)
\end{textcode}

Similar result with Octopus.

{\centering
\includegraphics[scale=1.0]{../codes/ks_dft_02/LOG_files/IMG_LiH.pdf}
}



\subsection{$\ce{CH4}$ molecule}

\begin{textcode}
   44      -8.0428330362   4.9289199566e-07   3.3001772073e-09
   45      -8.0428321264   9.0977377454e-07   2.8064033639e-09

SCF is converged in iter: 45

Eigenvalues:
  1      -0.6243171116
  2      -0.3496085566
  3      -0.3484490903
  4      -0.3424495986
----------------------------
Total energy components
----------------------------
Kinetic =       6.6836723909
Ps_loc  =     -36.7944583474
Ps_nloc =       0.4319519028
Hartree =      15.1969310575
XC      =      -3.0932073934
NN      =       9.5322782633
----------------------------
Total   =      -8.0428321264 
\end{textcode}

\begin{textcode}
 *********************** SCF CYCLE ITER #   13 ************************
 etot  = -8.06433473E+00 abs_ev   =  2.54E-05 rel_ev   =  7.64E-06
 ediff =       -9.29E-11 abs_dens =  7.73E-05 rel_dens =  9.66E-06
Matrix vector products:     36
Converged eigenvectors:      0

#  State  Eigenvalue [H]  Occupation    Error
      1       -0.624282    2.000000   (2.3E-06)
      2       -0.350544    2.000000   (1.0E-06)
      3       -0.348888    2.000000   (6.2E-07)
      4       -0.341658    2.000000   (2.0E-06)   
\end{textcode}

{\centering
\includegraphics[scale=1.0]{../codes/ks_dft_02/LOG_files/IMG_CH4.pdf}
}