\chapter{Kohn-Sham equation part II}
\label{chap:ks_part_2}

Using nonlocal potential (pseudopotential)

In this work, we use norm-conserving pseudopotentials proposed
by Goedecker-Teter-Hutter (GTH) \cite{Goedecker1996}.
These pseudopotentials can be written in terms of
local $V^{\mathrm{PS}}_{\mathrm{loc}}$ and
angular momentum $l$ dependent
nonlocal components $\Delta V^{\mathrm{PS}}_{l}$:
\begin{equation}
V_{\mathrm{ene-nuc}}(\mathbf{r}) =
\sum_{I} \left[
V^{\mathrm{PS}}_{\mathrm{loc}}(\mathbf{r}-\mathbf{R}_{I}) +
\sum_{l=0}^{l_{\mathrm{max}}}
V^{\mathrm{PS}}_{l}(\mathbf{r}-\mathrm{R}_{I},\mathbf{r}'-\mathbf{R}_{I})
\right]
\end{equation}
%
The local pseudopotential for
$I$-th atom, $V^{\mathrm{PS}}_{\mathrm{loc}}(\mathbf{r}-\mathbf{R}_{I})$,
is radially symmetric
function with the following radial form
\begin{equation}
V^{\mathrm{PS}}_{\mathrm{loc}}(r) =
-\frac{Z_{\mathrm{val}}}{r}\mathrm{erf}\left[
\frac{\bar{r}}{\sqrt{2}} \right] +
\exp\left[-\frac{1}{2}\bar{r}^2\right]\left(
C_{1} + C_{2}\bar{r}^2 + C_{3}\bar{r}^4 + C_{4}\bar{r}^6
\right)
\label{eq:V_ps_loc_R}
\end{equation}
with $\bar{r}=r/r_{\mathrm{loc}}$ and $r_{\mathrm{loc}}$, $Z_{\mathrm{val}}$,
$C_{1}$, $C_{2}$, $C_{3}$ and $C_{4}$ are the corresponding pseudopotential
parameters.
In $\mathbf{G}$-space, the GTH local pseudopotential can be written as
\begin{multline}
V^{\mathrm{PS}}_{\mathrm{loc}}(G) = -\frac{4\pi}{\Omega}\frac{Z_{\mathrm{val}}}{G^2}
\exp\left[-\frac{x^2}{2}\right] +
\sqrt{8\pi^3} \frac{r^{3}_{\mathrm{loc}}}{\Omega}\exp\left[-\frac{x^2}{2}\right]\times\\
\left( C_{1} + C_{2}(3 - x^2) + C_{3}(15 - 10x^2 + x^4) + C_{4}(105 - 105x^2 + 21x^4 - x^6) \right)
\label{eq:V_ps_loc_G}
\end{multline}
where $x=G r_{\mathrm{loc}}$.
%
The nonlocal component of GTH pseudopotential can written in real space as
\begin{equation}
V^{\mathrm{PS}}_{l}(\mathbf{r}-\mathbf{R}_{I},\mathbf{r}'-\mathbf{R}_{I}) =
\sum_{\mu=1}^{N_{l}} \sum_{\nu=1}^{N_{l}} \sum_{m=-l}^{l}
\beta_{\mu lm}(\mathbf{r}-\mathbf{R}_{I})\,
h^{l}_{\mu\nu}\,
\beta^{*}_{\nu lm}(\mathbf{r}'-\mathbf{R}_{I})
\end{equation}
where $\beta_{\mu lm}(\mathbf{r})$ are atomic-centered projector functions
\begin{equation}
\beta_{\mu lm}(\mathbf{r}) = 
p^{l}_{\mu}(r) Y_{lm}(\hat{\mathbf{r}})
\label{eq:proj_NL_R}
\end{equation}
%
and $h^{l}_{\mu\nu}$ are the pseudopotential parameters and
$Y_{lm}$ are the spherical harmonics. Number of projectors per angular
momentum $N_{l}$ may take value up to 3 projectors.
%
In $\mathbf{G}$-space, the nonlocal part of GTH pseudopotential can be described by
the following equation.
\begin{equation}
V^{\mathrm{PS}}_{l}(\mathbf{G},\mathbf{G}') =
(-1)^{l} \sum_{\mu}^{3} \sum_{\nu}^{3}\sum_{m=-l}^{l}
\beta_{\mu l m}(\mathbf{G}) h^{l}_{\mu\nu}
\beta^{*}_{\nu l m}(\mathbf{G}')
\end{equation}
with the projector functions
\begin{equation}
\beta_{\mu lm}(\mathbf{G}) = p^{l}_{\mu}(G) Y_{lm}(\hat{\mathbf{G}})
\label{eq:betaNL_G}
\end{equation}
The radial part of projector functions take the following form
\begin{equation}
p^{l}_{\mu}(G) = q^{l}_{\mu}\left(Gr_{l}\right)
\frac{\pi^{5/4}G^{l}\sqrt{ r_{l}^{2l+3}}}{\sqrt{\Omega}}
\exp\left[-\frac{1}{2}G^{2}r^{2}_{l}\right]
\label{eq:proj_NL_G}
\end{equation}
%
For $l=0$, we consider up to $N_{l}=3$ projectors:
\begin{align}
q^{0}_{1}(x) & = 4\sqrt{2} \\
q^{0}_{2}(x) & = 8\sqrt{\frac{2}{15}}(3 - x^2) \\
q^{0}_{3}(x) & = \frac{16}{3}\sqrt{\frac{2}{105}} (15 - 20x^2 + 4x^4)
\end{align}
%
For $l=1$, we consider up to $N_{l}=3$ projectors:
\begin{align}
q^{1}_{1}(x) & = 8 \sqrt{\frac{1}{3}} \\
q^{1}_{2}(x) & = 16 \sqrt{\frac{1}{105}} (5 - x^2) \\
q^{1}_{3}(x) & = 8 \sqrt{\frac{1}{1155}} (35 - 28x^2 + 4x^4)
\end{align}
%
For $l=2$, we consider up to $N_{l}=2$ projectors:
\begin{align}
q^{2}_{1}(x) & = 8\sqrt{\frac{2}{15}} \\
q^{2}_{2}(x) & = \frac{16}{3} \sqrt{\frac{2}{105}}(7 - x^2)
\end{align}
%
For $l=3$, we only consider up to $N_{l}=1$ projector:
\begin{equation}
q^{3}_{1}(x) = 16\sqrt{\frac{1}{105}}
\end{equation}

In the present implementation, we construct the local and nonlocal
components of pseudopotential in the $\mathbf{G}$-space using
their Fourier-transformed expressions
and transformed them back to real space if needed.
We refer the readers to the original
reference \cite{Goedecker1996} and the book \cite{Marx2009}
for more information about GTH pseudopotentials.

Due to the separation of local and non-local components of electrons-nuclei
interaction, Equation \eqref{eq:E_ele_nuc} can be written as
\begin{equation}
E_{\mathrm{ele-nuc}} = E^{\mathrm{PS}}_{\mathrm{loc}}
+ E^{\mathrm{PS}}_{\mathrm{nloc}}
\end{equation}
%
The local pseudopotential contribution is
\begin{equation}
E^{\mathrm{PS}}_{\mathrm{loc}} =
\int_{\Omega} \rho(\mathbf{r})\,V^{\mathrm{PS}}_{\mathrm{loc}}(\mathbf{r})\,
\mathrm{d}\mathbf{r}
\end{equation}
%
and the non-local contribution is
\begin{equation}
E^{\mathrm{PS}}_{\mathrm{nloc}} = 
\sum_{\mathbf{k}}
\sum_{i}
w_{\mathbf{k}}
f_{i\mathbf{k}}
\int_{\Omega}\,
\psi^{*}_{i\mathbf{k}}(\mathbf{r})
\left[
\sum_{I}\sum_{l=0}^{l_{\mathrm{max}}}
V^{\mathrm{PS}}_{l}(\mathbf{r}-\mathbf{R}_{I},\mathbf{r}'-\mathbf{R}_{I})
\right]
\psi_{i\mathbf{k}}(\mathbf{r})
\,\mathrm{d}\mathbf{r}.
\end{equation}