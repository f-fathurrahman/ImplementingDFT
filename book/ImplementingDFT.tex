%\documentclass[graybox,envcountchap,sectrefs,default]{svmono}
\documentclass[a4paper,11pt]{extbook}

\usepackage{amsmath}
\usepackage{amssymb}
\usepackage{graphicx}
\usepackage{float}

%\usepackage[T1]{fontenc}
%\usepackage[charter]{mathdesign}
%\usepackage[varg]{txfonts}

%\usepackage{geometry}
%\geometry{verbose,tmargin=1.5cm,bmargin=1.5cm,lmargin=1.5cm,rmargin=1.5cm}

\usepackage{geometry}
\geometry{verbose,tmargin=3cm,bmargin=3cm,lmargin=3cm,rmargin=3cm}

% DRAFT SETTINGS: easier to read in dark mode
%\usepackage{cmbright}
%\renewcommand{\familydefault}{\sfdefault}

\usepackage[libertine]{newtxmath}
\usepackage[no-math]{fontspec}
\setmainfont{Linux Libertine O}

%\usepackage{fontspec}
%\setmainfont{FreeSerif}
%\setmonofont{FreeMono}
%\setmonofont{DejaVu Sans Mono}
\setmonofont{JuliaMono-Regular}

\usepackage{framed}

\usepackage{braket}
\usepackage{mhchem}

\setcounter{secnumdepth}{3}
\setcounter{tocdepth}{3}
\setlength{\parskip}{\smallskipamount}
\setlength{\parindent}{0pt}
\usepackage{setspace}
\onehalfspacing

\usepackage{minted}
% using DejaVu Sans Mono, we need to scale the font to smaller size to match
% the main font.
\newminted{julia}{breaklines,fontsize=\scriptsize}
\newminted{bash}{breaklines,fontsize=\scriptsize}
\newminted{text}{breaklines,fontsize=\scriptsize}

\newcommand{\txtinline}[1]{\mintinline[fontsize=\scriptsize]{text}{#1}}
\newcommand{\jlinline}[1]{\mintinline[fontsize=\scriptsize]{julia}{#1}}

\newmintedfile[juliafile]{julia}{breaklines,fontsize=\scriptsize}

\usepackage{float}
\newfloat{mintedfloat}{h}{lop}
\floatname{mintedfloat}{Listing}

% Using background color for minted environment
\usepackage{xcolor}
\definecolor{mintedbg}{rgb}{0.9,0.9,0.9}
\usepackage{mdframed}
\BeforeBeginEnvironment{minted}{
    \begin{mdframed}[backgroundcolor=mintedbg,%
        topline=false,bottomline=false,%
        leftline=false,rightline=false]
}
\AfterEndEnvironment{minted}{\end{mdframed}}

\usepackage{hyperref}

\usepackage{multicol}

\usepackage{makeidx}
\makeindex

\begin{document}

\title{Implementing Density Functional Theory Calculations}
\author{Fadjar Fathurrahman \\
Hermawan Kresno Dipojono}
\maketitle

\frontmatter

%\begin{dedication}

Untuk Mariya Al Qibtiya Nasution, istriku tercintah {\color{red}$\heartsuit$}

\end{dedication}





\include{foreword}
\preface

Importance of density functional theory

Implementation of density functional theory in various program packages (free and commercial)

Several books about density functional theories

The problem: not yet giving necessary details

This book is our humble attempt to demystifying several aspects of practical density functional
theory to beginners in the field.

Objective of this book: show the reader how to implement a density functional theory
for simple system containing only model potential (such as harmonic potential)
and to non-local pseudopotentials which are usually used in typical DFT calculations
for molecular and crystalline systems.

Outline of the book:
1d, 2d, 3d, Schrodinger equation, Poisson equation, Kohn-Sham equation for local
(pseudo)potentials, Kohn-Sham equation for nonlocal pseudopotentials.

This is for \textit{acknowledgments}.
 
\vspace{\baselineskip}
\begin{flushright}\noindent
Bandung,\hfill {\it Fadjar Fathurrahman}\\
month year\hfill {\it Hermawan Kresno Dipojono}\\
\end{flushright}



%\include{acknowledgement}

\tableofcontents

\mainmatter

\chapter{An introduction to density functional theory}

Intro to DFT \cite{Kohn1965}.

Kohn-Sham equation \index{Kohn-Sham equation}
\begin{equation}
\nabla^2 + V(\mathbf{r}) \psi_{i}\mathbf{r} = E\,\psi_{i}(\mathbf{r})
\end{equation}


\begin{thebibliography}{99.}%
\bibitem{Kohn1965} Test Kohn Sham
\end{thebibliography}
\chapter{Schroedinger equation in 1d}

We are interested in finding bound states solution to 1d time-independent Schroedinger equation:
\begin{equation}
\left[ -\frac{1}{2}\frac{\mathrm{d}^2}{\mathrm{d}x^2} + V(x) \right] \psi(x) = E\, \psi(x)
\label{eq:Sch_1d_eq}
\end{equation}
%
with the boundary conditions:
%
\begin{equation}
\lim_{x \rightarrow \pm \infty} \psi(x) = 0
\label{eq:BC_isolated}
\end{equation}
%
First we need to define a spatial domain $\left[x_{\mathrm{min}}, x_{\mathrm{max}}\right]$
where $x_{\mathrm{min}}, x_{\mathrm{max}}$ chosen
such that the boundary condition \ref{eq:BC_isolated} is approximately satisfied.
The next step is to divide the spatial domain $x$ using equally-spaced grid points
which we will denote as $\{x_{1},x_{2},\ldots,x_{N}\}$ where $N$ is number
of grid points. Various spatial quantities such as wave function and potential will be discretized
on these grid points.
The grid points $x_{i}$, $i = 1, 2, \ldots$ are chosen as:
\begin{equation}
x_{i} = x_{\mathrm{min}} + (i-1)h
\end{equation}
where $h$ is the spacing between the grid points:
\begin{equation}
h = \frac{ x_{\mathrm{max}} - x_{\mathrm{min}} }{N-1}
\end{equation}

The following code can be used to initialize the grid points:
\begin{juliacode}
function init_FD1d_grid( x_min::Float64, x_max::Float64, N::Int64 )
    L = x_max - x_min
    h = L/(N-1) # spacing
    x = zeros(Float64,N) # the grid points
    for i = 1:N
        x[i] = x_min + (i-1)*h
    end
    return x, h
end
\end{juliacode}


\section{Approximating second derivative}

Our next task is to find an approximation to the second derivative operator
present in the Equation \eqref{eq:Sch_1d_eq}.
One simple approximation that we can use is the 3-point (central) finite difference:
\begin{equation}
\frac{\mathrm{d}^2}{\mathrm{d}x^2} \psi_{i} =
\frac{\psi_{i+1} - 2\psi_{i} + \psi_{i-1}}{h^2}
\end{equation}
where we have the following notation have been used: $\psi_{i} = \psi(x_{i})$.
%
By taking $\{ \psi_{i} \}$ as a column vector, the second derivative operation
can be expressed as matrix multiplication:
\begin{equation}
\vec{\psi''} = \mathbb{D}^{(2)} \vec{\psi}
\end{equation}
%%
where $\mathbb{D}^{(2)}$ is the second derivative matrix operator:
\begin{equation}
\mathbb{D}^{(2)} = \frac{1}{h^2}
\begin{bmatrix}
-2  &  1  &  0  &  0  & 0 & \cdots & 0 \\
 1  & -2  &  1  &  0  & 0 & \cdots & 0 \\
 0  &  1  & -2  &  1  & 0 & \cdots & 0 \\
 \vdots  &  \ddots  &  \ddots  & \ddots  & \ddots  & \ddots & \vdots \\
 0 & \cdots & 0 & 1 & -2 & 1 & 0 \\
 0  &  \cdots  & \cdots & 0  & 1  & -2  & 1 \\
 0  &  \cdots  & \cdots & \cdots & 0  &  1  & -2 \\
\end{bmatrix}
\label{eq:1d_D2_matmul}
\end{equation}

An example implementation can be found in the following function.
\begin{juliacode}
function build_D2_matrix_3pt( N::Int64, h::Float64 )
    mat = zeros(Float64,N,N)
    for i = 1:N-1
        mat[i,i] = -2.0
        mat[i,i+1] = 1.0
        mat[i+1,i] = mat[i,i+1]
    end
    mat[N,N] = -2.0
    return mat/h^2
end
\end{juliacode}


Before use this function to solve Schroedinger equation we will to test the operation
in Equation \eqref{eq:1d_D2_matmul} for a simple function which second derivative
can be calculated analytically.
\begin{equation}
\psi(x) = \mathrm{e}^{-\alpha x^2}
\end{equation}
%
which second derivative can be calculated as
%
\begin{equation}
\psi''(x) = \left( -2 \alpha + 4\alpha^2 x^2 \right) \mathrm{e}^{-\alpha x^2}
\end{equation}
%
They are implemented in the following code
\begin{juliacode}
function my_gaussian(x; α=1.0)
    return exp(-α*x^2)
end

function d2_my_gaussian(x; α=1.0)
    return (-2*α + 4*α^2 * x^2) * exp(-α*x^2)
end
\end{juliacode}

\begin{figure}[H]
{\center
\includegraphics[scale=0.75]{../codes/1d/IMG_gaussian_15.pdf}
\includegraphics[scale=0.75]{../codes/1d/IMG_gaussian_51.pdf}
\par}
\caption{Finite difference approximation to a Gaussian function and its second derivative}
\end{figure}


\section{Harmonic potential}

We will start with a simple potential with known exact solution, namely the harmonic potential:
\begin{equation}
V(x) = \frac{1}{2}\omega^2 x^2
\end{equation}

The Hamiltonian in finite difference representation:
\begin{equation}
\mathbb{H} = -\frac{1}{2}\mathbb{D}^{(2)} + \mathbb{V}
\end{equation}
where $\mathbb{V}$ is a diagonal matrix whose elements are:
\begin{equation}
\mathbb{V}_{ij} = V(x_{i})\delta_{ij}
\end{equation}


Code to solve harmonic oscillator:

\juliafile{../codes/1d/main_harmonic_01.jl}

Compare with analytical solution.

Plot of eigenfunctions:

\begin{figure}[H]
{\center
\includegraphics[scale=0.75]{../codes/1d/IMG_main_harmonic_01_51.pdf}
\par}
\caption{Eigenstates of harmonic oscillator}
\end{figure}


\section{Higher order finite difference}

To obtain higher accuracy

Implementing higher order finite difference.


\section{Exercises}

Gaussian potential

\chapter{Schroedinger equation in 2d}

Schrodinger equation in 2d:
\begin{equation}
\left[ -\frac{1}{2}\nabla^2 + V(x,y) \right] \psi(x,y) = E\,\psi(x,y)
\end{equation}
%
where $\nabla^2$ is the Laplacian operator:
\begin{equation}
\nabla^2 = \frac{\partial^2}{\partial x^2} + \frac{\partial^2}{\partial y^2}
\end{equation}


\section{Finite difference grid in 2d}

\begin{juliacode}
struct FD2dGrid
    Npoints::Int64
    Nx::Int64
    Ny::Int64
    hx::Float64
    hy::Float64
    dA::Float64
    x::Array{Float64,1}
    y::Array{Float64,1}
    r::Array{Float64,2}
    idx_ip2xy::Array{Int64,2}
    idx_xy2ip::Array{Int64,2}
end
\end{juliacode}


\begin{juliacode}
function FD2dGrid( x_domain, Nx, y_domain, Ny )
    x, hx = init_FD1d_grid(x_domain, Nx)
    y, hy = init_FD1d_grid(y_domain, Ny)
    dA = hx*hy
    Npoints = Nx*Ny
    r = zeros(2,Npoints)
    ip = 0
    idx_ip2xy = zeros(Int64,2,Npoints)
    idx_xy2ip = zeros(Int64,Nx,Ny)
    for j in 1:Ny
        for i in 1:Nx
            ip = ip + 1
            r[1,ip] = x[i]
            r[2,ip] = y[j]
            idx_ip2xy[1,ip] = i
            idx_ip2xy[2,ip] = j
            idx_xy2ip[i,j] = ip
        end
    end
    return FD2dGrid(Npoints, Nx, Ny, hx, hy, dA, x, y, r, idx_ip2xy, idx_xy2ip)
end
\end{juliacode}

\section{Laplacian operator}

Given second derivative matrix in $x$, $\mathbb{D}^{(2)}_{x}$,
$y$ direction, $\mathbb{D}^{(2)}_{x}$,
we can construct finite difference representation of the Laplacian operator
$\mathbb{L}$ by using
%
\begin{equation}
\mathbb{L} = \mathbb{D}^{(2)}_{x} \otimes \mathbb{I}_{y} +
\mathbb{I}_{x} \otimes \mathbb{D}^{(2)}_{y}
\end{equation}
%
where $\otimes$ is Kronecker product.
In Julia, we can use the function \jlinline{kron} to form the Kronecker product
between two matrices \jlinline{A} and \jlinline{B} as \jlinline{kron(A,B)}.

\begin{juliacode}
function build_nabla2_matrix( fdgrid::FD2dGrid; func_1d=build_D2_matrix_3pt )
    Nx = fdgrid.Nx
    hx = fdgrid.hx
    Ny = fdgrid.Ny
    hy = fdgrid.hy

    D2x = func_1d(Nx, hx)
    D2y = func_1d(Ny, hy)

    ∇2 = kron(D2x, speye(Ny)) + kron(speye(Nx), D2y)
    return ∇2
end
\end{juliacode}

Example to the approximation of 2nd derivative of 2d Gaussian function

\begin{figure}[H]
{\center
\includegraphics[width=0.45\textwidth]{../codes/FD2d/IMG_gaussian2d.pdf}
\includegraphics[width=0.45\textwidth]{../codes/FD2d/IMG_d2_gaussian2d.pdf}
\par}
\caption{Two-dimensional Gaussian function and its finite difference
approximation of second derivative}
\end{figure}

\section{Iterative methods for eigenvalue problem}

The Hamiltonian matrix:
\begin{juliacode}
∇2 = build_nabla2_matrix( fdgrid, func_1d=build_D2_matrix_9pt )
Ham = -0.5*∇2 + spdiagm( 0 => Vpot )
\end{juliacode}

The Hamiltonian matrix size is large. The use \jlinline{eigen} method
to solve this eigenvalue problem is not practical.
We also do not need to solve for all eigenvalues.
We must resort to the
so called iterative methods.

\chapter{Schroedinger equation in 3d}

After we have considered two-dimensional Schroedinger equations, we are now ready for
the extension to three-dimensional systems. In 3d, Schroedinger equation can be
written as:
\begin{equation}
\left[ -\frac{1}{2}\nabla^2 + V(\mathbf{r}) \right] \psi(\mathbf{r}) = E\,\psi(\mathbf{r})
\end{equation}
where $\mathbf{r}$ is the abbreviation to $(x,y,z)$ and
%
$\nabla^2$ is the Laplacian operator in 3d:
\begin{equation}
\nabla^2 = \frac{\partial^2}{\partial x^2} + \frac{\partial^2}{\partial y^2} +
\frac{\partial^2}{\partial z^2}
\end{equation}

\subsection{Three-dimensional grid}

As in the preceeding chapter, our first task is to create a representation of 3d grid
points and various quantities defined on it. This task is realized using straightforward
extension of \txtinline{FD2dGrid} to \txtinline{FD3dGrid}.


Visualization of 3d functions as isosurface map or slice of 3d array.

Introducing 3d xsf


\subsection{Laplacian operator}

\begin{equation}
\mathbb{L} = \mathbb{D}^{(2)}_{x} \otimes \mathbb{I}_{y} \otimes \mathbb{I}_{z} +
\mathbb{I}_{x} \otimes \mathbb{D}^{(2)}_{y} \otimes \mathbb{I}_{z} +
\mathbb{I}_{x} \otimes \mathbb{I}_{y} \otimes \mathbb{D}^{(2)}_{z}
\end{equation}


Code
\begin{juliacode}
const ⊗ = kron
function build_nabla2_matrix( fdgrid::FD3dGrid; func_1d=build_D2_matrix_3pt )
    D2x = func_1d(fdgrid.Nx, fdgrid.hx)
    D2y = func_1d(fdgrid.Ny, fdgrid.hy)
    D2z = func_1d(fdgrid.Nz, fdgrid.hz)
    IIx = speye(fdgrid.Nx)
    IIy = speye(fdgrid.Ny)
    IIz = speye(fdgrid.Nz)
    ∇2 = D2x⊗IIy⊗IIz + IIx⊗D2y⊗IIz + IIx⊗IIy⊗D2z 
    return ∇2
end
\end{juliacode}


\chapter{Poisson equation}
\label{chap:poisson_3d}

\section{Conjugate gradient method}

In this section we will discuss a second type of equation
that is important in solving Kohn-Sham equation,
namely the Poisson equation. In the
context of solving Kohn-Sham equation, Poisson equation is used to
calculate classical electrostatic potential due to some electronic
charge density.
The Poisson equation that we will solve have the following form:
\begin{equation}
\nabla^2 V_{\mathrm{Ha}}(\mathbf{r}) = -4\pi\rho(\mathbf{r})
\label{eq:poisson_3d}
\end{equation}
where $\rho(\mathbf{r})$ is the electronic density. Using finite
difference discretization for the operator $\nabla^2$ we end up with
the following linear equation:
\begin{equation}
\mathbb{L} \mathbf{V} = \mathbf{f}
\label{eq:linear_eq_poisson}
\end{equation}
where $\mathbb{L}$ is the matrix representation of the Laplacian operator
$\mathbf{f}$ is the discrete representation of the right hand side of the equation
\ref{eq:poisson_3d}, and the unknown $\mathbf{V}$ is the discrete representation of
the Hartree potential.

There exist several methods for solving the linear equation \ref{eq:linear_eq_poisson}.
We will use the so-called conjugate gradient method for solving this equation.
This method is an iterative method, so it generally needs a good preconditioner to
achieve good convergence. A detailed derivation about the algorithm is beyond this
article and the readers are referred to several existing literatures \cite{Hestenes1952,Shewchuk1994}
and a webpage \cite{wiki-Conjugate-gradient} for more
information. The algorithm is described in \txtinline{Poisson_solve_PCG.jl}

\begin{juliacode}
function Poisson_solve_PCG( Lmat, prec, f; NiterMax=1000 TOL=5.e-10 )
  Npoints = size(f,1)
  phi = zeros( Float64, Npoints )
  r = zeros( Float64, Npoints )
  p = zeros( Float64, Npoints )
  z = zeros( Float64, Npoints )
  nabla2_phi = Lmat*phi
  r = f - nabla2_phi
  z = copy(r)
  ldiv!(prec, z)
  p = copy(z)
  rsold = dot( r, z )
  for iter = 1 : NiterMax
    nabla2_phi = Lmat*p
    alpha = rsold/dot( p, nabla2_phi )
    phi = phi + alpha * p
    r = r - alpha * nabla2_phi
    z = copy(r)
    ldiv!(prec, z)
    rsnew = dot(z, r)
    deltars = rsold - rsnew
    if sqrt(abs(rsnew)) < TOL
      break
    end
    p = z + (rsnew/rsold) * p
    rsold = rsnew
  end
  return phi
end
\end{juliacode}

To test our implementation we will adopt a problem given in Prof. Arias Practical
DFT mini-course \cite{practical-DFT-mini-course}.
In this problem we will solve Poisson equation for a given charge density built from
superposition of two Gaussian charge density:
\begin{equation}
\rho(\mathbf{r}) = \frac{1}{(2\pi\sigma_{1}^{2})^{\frac{3}{2}}} \exp\left( -\frac{\mathbf{r}^2}{2\sigma_{1}^{2}} \right)
- \frac{1}{(2\pi\sigma_{2}^{2})^{\frac{3}{2}}} \exp\left( -\frac{\mathbf{r}^2}{2\sigma_{2}^{2}} \right)
\end{equation}
After we obtain $V_{\mathrm{Ha}}(\mathbf{r})$, we calculate the Hartree energy:
\begin{equation}
E_{\mathrm{Ha}} = \frac{1}{2} \int \rho(\mathbf{r}) V_{\mathrm{Ha}}(\mathbf{r})\,\mathrm{d}\mathbf{r}
\end{equation}
and compare the result with the analytical formula.

\begin{juliacode}
function test_main( NN::Array{Int64} )
  AA = [0.0, 0.0, 0.0]
  BB = [16.0, 16.0, 16.0]
  # Initialize grid
  FD = FD3dGrid( NN, AA, BB )
  # Box dimensions
  Lx = BB[1] - AA[1]
  Ly = BB[2] - AA[2]
  Lz = BB[3] - AA[3]
  # Center of the box
  x0 = Lx/2.0
  y0 = Ly/2.0
  z0 = Lz/2.0
  # Parameters for two gaussian functions
  sigma1 = 0.75 
  sigma2 = 0.50

  Npoints = FD.Nx * FD.Ny * FD.Nz
  rho = zeros(Float64, Npoints)
  phi = zeros(Float64, Npoints)
  # Initialization of charge density
  dr = zeros(Float64,3)
  for ip in 1:Npoints
    dr[1] = FD.r[1,ip] - x0
    dr[2] = FD.r[2,ip] - y0
    dr[3] = FD.r[3,ip] - z0
    r = norm(dr)
    rho[ip] = exp( -r^2 / (2.0*sigma2^2) ) / (2.0*pi*sigma2^2)^1.5 -
              exp( -r^2 / (2.0*sigma1^2) ) / (2.0*pi*sigma1^2)^1.5
  end
  deltaV = FD.hx * FD.hy * FD.hz
  Laplacian3d = build_nabla2_matrix( FD, func_1d=build_D2_matrix_9pt )
  prec = aspreconditioner(ruge_stuben(Laplacian3d))
  @printf("Test norm charge: %18.10f\n", sum(rho)*deltaV)
  print("Solving Poisson equation:\n")
  phi = Poisson_solve_PCG( Laplacian3d, prec, -4*pi*rho, 1000, verbose=true, TOL=1e-10 )
  # Calculation of Hartree energy
  Unum = 0.5*sum( rho .* phi ) * deltaV
  Uana = ((1.0/sigma1 + 1.0/sigma2 )/2.0 - sqrt(2.0)/sqrt(sigma1^2 + sigma2^2))/sqrt(pi)
  @printf("Numeric  = %18.10f\n", Unum)
  @printf("Uana     = %18.10f\n", Uana)
  @printf("abs diff = %18.10e\n", abs(Unum-Uana))
end
test_main([64,64,64])
\end{juliacode}

Result:
\begin{textcode}
Numeric  =       0.0551434259
Uana     =       0.0551425277
abs diff =   8.9818466372e-07
\end{textcode}

FIXME: Needs correction, boundary condition is not treated properly.

\section{Reciprocal space method}

Another popular method for solving Poisson equation is by using
reciprocal space method.
This method is the natural choice for systems with periodic boundary condition.

In the reciprocal or $\mathbf{G}$-space, Poisson equation is transformed to
\begin{equation}
\mathbf{G}^{2} V_{\mathrm{Ha}}(\mathbf{G}) = 4\pi\rho(\mathbf{G})
\end{equation}
for which we can solve:
\begin{equation}
V_{\mathrm{Ha}}(\mathbf{G}) = 4\pi \frac{\rho(\mathbf{G})}{\mathbf{G}^{2}}
\label{eq:V_Ha_G}
\end{equation}
Note that, we must exclude the term $\mathbf{G}=\mathbf{0}$ in the Equation
\ref{eq:V_Ha_G}.

TODO: Move discussion about GVectors to Chapter about 3d Sch equation?
(periodic system)

The $\mathbf{G}$-vectors are defined as
\begin{equation}
\mathbf{G} = n_{1}\mathbf{b}_{1} + n_{2}\mathbf{b}_{2} + n_{3}\mathbf{b}_{3}
\end{equation}
where $b_{i} = \frac{2\pi}{a_{i}}$.

Initializing GVectors:
\begin{juliacode}
function GVectors( grid )
  Npoints = grid.Npoints
  Nx = grid.Nx; Ny = grid.Ny; Nz = grid.Nz
  Lx = grid.Lx; Ly = grid.Ly; Lz = grid.Lz
  RecVecs = zeros(3,3)
  RecVecs[1,1] = 2.0*pi/Lx
  RecVecs[2,2] = 2.0*pi/Ly
  RecVecs[3,3] = 2.0*pi/Lz
  Δ = max(grid.hx, grid.hy, grid.hz)
  ecutrho = (pi/Δ)^2
  Ns = (Nx,Ny,Nz)
  Ng = calc_Ng( Ns, RecVecs, ecutrho )
  G  = zeros(Float64,3,Ng)
  G2 = zeros(Float64,Ng)
  idx_g2r = zeros(Int64,Ng)
  ig = 0
  ip = 0
  for k in 0:Ns[3]-1, j in 0:Ns[2]-1, i in 0:Ns[1]-1
      ip = ip + 1
      gi = _flip_fft( i, Ns[1] )
      gj = _flip_fft( j, Ns[2] )
      gk = _flip_fft( k, Ns[3] )
      Gx = RecVecs[1,1]*gi + RecVecs[1,2]*gj + RecVecs[1,3]*gk
      Gy = RecVecs[2,1]*gi + RecVecs[2,2]*gj + RecVecs[2,3]*gk
      Gz = RecVecs[3,1]*gi + RecVecs[3,2]*gj + RecVecs[3,3]*gk
      G2_temp = Gx^2 + Gy^2 + Gz^2
      if 0.5*G2_temp <= ecutrho
          ig = ig + 1
          G[1,ig] = Gx
          G[2,ig] = Gy
          G[3,ig] = Gz
          G2[ig] = G2_temp
          idx_g2r[ig] = ip
      end
  end
  idx_sorted = sortperm(G2)
  G = G[:,idx_sorted]
  G2 = G2[idx_sorted]
  idx_g2r = idx_g2r[idx_sorted]
  G2_shells, idx_g2shells = init_Gshells( G2 )
  return GVectors(Ng, G, G2, idx_g2r, G2_shells, idx_g2shells)
end
\end{juliacode}

Implementation:
\begin{juliacode}
function Poisson_solve_fft( grid, gvec::GVectors, rho::Vector{Float64} )
  Npoints = grid.Npoints
  Nx = grid.Nx; Ny = grid.Ny; Nz = grid.Nz
  ctmp = zeros(ComplexF64,Nx,Ny,Nz)
  ctmp[:] = rho[:]
  # to reciprocal space
  fft!(ctmp)
  ctmp[1] = 0.0 + im*0.0
  for ig in 2:gvec.Ng
      ip = gvec.idx_g2r[ig]
      ctmp[ip] = 4.0*pi*ctmp[ip]/gvec.G2[ig]
  end
  # to real space
  ifft!(ctmp)
  return reshape(real(ctmp),Npoints)
end
\end{juliacode}



\section{Direct integration}

An alternative to solving Poisson equation is to integrate the equation defining
the Hartree potential directly. However a naive integration algorithm will scale
badly. In this section, we will adopt a direct integration method that is proposed
in \cite{Sundholm2005}. We will only describe the method for isolated system.
The extension to periodic system is described in \cite{Losilla2010}.

We will begin from the definition of Hartree potential:
\begin{equation}
V(\mathbf{r}_{1}) = \int_{-\infty}^{+\infty}
\rho(\mathbf{r}_{2}) \frac{1}{r_{12}} \, \mathrm{d}\mathbf{r}_{2}
\end{equation}

The Coulomb operator, $\frac{1}{r}$ can we written as the integral
\begin{equation}
\frac{1}{r} = \frac{2}{\sqrt{\pi}} \int_{0}^{\infty} e^{-r^2 t^2}\,\mathrm{d}t
\end{equation}
Using this identity, the Hartree potential is written as
\begin{equation}
V(\mathbf{r}_{1}) = \frac{2}{\sqrt{\pi}}
\int_{0}^{\infty}
\int_{-\infty}^{+\infty}
e^{-t^2(\mathbf{r}_1 - \mathbf{r}_2)^2} \rho(\mathbf{r}_2)
\, \mathrm{d}\mathbf{r}_{2}\mathrm{d}t
\end{equation}

By expanding the electron density in the following form
\begin{equation}
\rho(\mathbf{r}_2) = \sum_{\alpha\beta\gamma} d_{\alpha\beta\gamma}
\chi_{\alpha}(x_2) \chi_{\beta}(y_2) \chi_{\gamma}(z_2)
\end{equation}
the Hartree potential can be written as
\begin{multline}
V_{0}(x_{1},y_{1},z_{1}) = \frac{2}{\sqrt{\pi}}\sum_{\alpha_t} \omega_{\alpha_t}
\sum_{\alpha\beta\gamma} d_{\alpha\beta\gamma}
\int_{-\infty}^{\infty} e^{-t^2_{\alpha_t}(z_{1} - z_{2}) } \chi_{\gamma}(z_2)\mathrm{d}z_2
\times \int_{-\infty}^{\infty} e^{-t^2_{\alpha_t}(y_{1} - y_{2}) } \chi_{\beta}(y_2)\mathrm{d}y_2 \\
\times \int_{-\infty}^{\infty} e^{-t^2_{\alpha_t}(x_{1} - x_{2}) } \chi_{\alpha}(x_2)\mathrm{d}x_2
\end{multline}
where
$t_{\alpha_t}$ are integration points and $w_{\alpha_t}$ are the corresponding
integration weights.
By defining the following quantity:
\begin{equation}
F_{\gamma_x \alpha_t}^{x,\alpha_x} = \int_{-\infty}^{\infty}
e^{-t^2_{\alpha_t} (x_{\alpha_x} - x_2)^2} \chi_{\gamma_x}(x_2)\,\mathrm{d}x_2
\end{equation}
Hartree potential can be written as
\begin{equation}
V_{\alpha_x \alpha_y \alpha_z} = \frac{2}{\sqrt{\pi}} \sum_{\alpha_t} w_{\alpha_t}
\sum_{\gamma_z} F_{\gamma_z \alpha_z}^{z,\alpha_t}
\sum_{\gamma_y} F_{\gamma_y \alpha_y}^{y,\alpha_t}
\sum_{\gamma_x} F_{\gamma_x \alpha_x}^{x,\alpha_t}
d_{\gamma_x \gamma_y \gamma_z}
\end{equation}



\chapter{Kohn-Sham equation part I}

In this chapter we will put together Schroedinger and Poisson solver
that we have built in the previous chapters to solve the Kohn-Sham equation:
\begin{equation}
\left[ -\frac{1}{2}\nabla^2 + V_{\mathrm{KS}}(\mathbf{r}) \right]
\psi_{i}(\mathbf{r}) = \epsilon_{i} \psi_{i}(\mathbf{r})
\end{equation}
where $V_{\mathrm{KS}}(\mathbf{r})$ is effective single particle potential or
the Kohn-Sham potential:
\begin{equation}
V_{\mathrm{KS}}(\mathbf{r}) =
V_{\mathrm{ext}}(\mathbf{r}) + V_{\mathrm{Ha}}(\mathbf{r}) + V_{\mathrm{xc}}(\mathbf{r})
\label{eq:KS_pot_local}
\end{equation}

The Kohn-Sham equation looks very much like Schroedinger equation,
but with two additional potentials: the Hartree and XC potential.
Hartree potential is the classical electrostatic interaction potential which is defined as
\begin{equation}
V_{\mathrm{Ha}}(\mathbf{r}) = \int \frac{\rho(\mathbf{r}')}{\left| \mathbf{r} - \mathbf{r}' \right|}
\,\mathrm{d}\mathbf{r}'
\end{equation}
where $\rho(\mathbf{r})$ is the electron density:
\begin{equation}
\rho(\mathbf{r}) = \sum_{i} f_{i} \psi_{i}^{*}(\mathbf{r}) \psi_{i}(\mathbf{r})
\label{eq:elec_dens_01}
\end{equation}
we can use any methods in Chapter \ref{chap:poisson_3d} to calculate
$V_{\mathrm{Ha}}(\mathbf{r})$. The factor $f_{i}$ in the Equation \ref{eq:elec_dens_01}
is the occupation number of the electron.
Exchange-correlation potential, $V_{\mathrm{xc}}$, has no explicit form in terms of
electron density. For practical purpose, we must resort to approximate forms which
will be given later.

Note that to calculate electron density, we must know the Kohn-Sham orbitals.
To obtain Kohn-Sham orbitals, we must solve the Kohn-Sham equations which
demand us to calculate $V_{\mathrm{Ha}}$ and $V_{\mathrm{xc}}$. Meanwhile,
we can not these potentials without knowing electron density.
Because of this chicken-and-egg
characteristic, the Kohn-Sham equation must be solved using self-consitent field (SCF) method.
We usually start from a guess or "input" electron density and solve the Kohn-Sham equation
for this guess density. From the solution of the Kohn-Sham equation, we obtain
Kohn-Sham energies and orbitals which can be used to calculate "output" electron density.
This "output" density is then compared with the "input" electron density. If they are
not the same then we create a new input density to the Kohn-Sham solver. This procedure
is repeated until the difference between "input" and "output" density is small.
In this case, we said that the SCF is convergent.

We also can monitor the convergence of other quantities other than electron density.
In this book, we will usually use total energy as the convergence criteria for SCF.
Thus, we will begin by reviewing total energy terms in a typical density functional
theory calculations.

\section{Total energy terms}

The main quantity of interest is total energy
\begin{equation}
E_{\mathrm{KS}} = E_{\mathrm{kin}} + E_{\mathrm{ext}}
+ E_{\mathrm{Ha}} + E_{\mathrm{xc}}
\end{equation}
%
The kinetic energy of noninteracting electrons:
\begin{equation}
E_{\mathrm{kin}} = -\frac{1}{2}\sum_{i} \int \psi_{i}^{*}(\mathbf{r}) \nabla^2 \psi_{i}(\mathbf{r})
\,\mathrm{d}\mathbf{r}
\end{equation}
%
External potential energy:
\begin{equation}
E_{\mathrm{ext}} = \int \rho(\mathbf{r}) V_{\mathrm{ext}}(\mathbf{r})
\,\mathrm{d}\mathbf{r}
\end{equation}
%
Hartree energy:
\begin{equation}
E_{\mathrm{Ha}} = \frac{1}{2} \int \rho(\mathbf{r}) V_{\mathrm{Ha}}(\mathbf{r})
\,\mathrm{d}\mathbf{r}
\end{equation}
%
XC energy (LDA):
\begin{equation}
E_{\mathrm{ext}} = \int \varepsilon_{\mathrm{xc}}[\rho(\mathbf{r})] \rho(\mathbf{r})
\,\mathrm{d}\mathbf{r}
\end{equation}
where $\varepsilon_{\mathrm{xc}}$ is the XC energy per particle per volume.

Alternative expression of total energy as sum of orbital energies:
\begin{equation}
E_{\mathrm{KS}} = \sum_{i}^{N} \epsilon_{i} - E_{\mathrm{Ha}} + E_{\mathrm{xc}}
- \int \frac{\delta E_{\mathrm{xc}}}{\delta \rho(\mathbf{r})} \rho(\mathbf{r})
\,\mathrm{d}\mathbf{r}
\end{equation}

Nucleus-nucleus interaction energy is also usually added to the Kohn-Sham energy
making up the total energy of the system:
\begin{equation}
E_{\mathrm{tot}} = E_{\mathrm{KS}} + E_{\mathrm{NN}}
\end{equation}
where:
\begin{equation}
E_{\mathrm{NN}} = \frac{1}{2} \sum_{I,J \neq I}
\frac{Z_{I} Z_{J}}{\left| \mathbf{R}_{I} - \mathbf{R}_{J} \right|}
\end{equation}



\section{Data structures}

We will first describe several custom data structure (or \jlinline{struct}s)
to make our program somewhat more manageable.

In the following, we will first consider the case of
$V_{\mathrm{xc}}=0$. This will make our first implementation
of self-consistent field to be easier.
We will consider a 3d systems, so the programs
that we have made in Chapter \ref{chap:sch_3d} will be used as the starting point
for building and Kohn-Sham solver.

\subsection{The Hamiltonian}

Because our Hamiltonian now contains additional potentials other that external ionic
potential, its action or multiplication to a wave function is
more complicated. We also need to implement several operations such as calculation
of electron density for given wave functions and total energy.
These calculations usually need to access to several variables. To make access to various
variables easy, we will wrap them in a "big" Julia struct. We will call this struct
as \jlinline{Hamiltonian}.

The definition of the \jlinline{Hamiltonian} struct can be seen in the following
Julia code.
\begin{fullwidth}
\begin{juliacode}
const AMG_PREC_TYPE = typeof( aspreconditioner(ruge_stuben(speye(1))) )
mutable struct Hamiltonian
    grid::Union{FD3dGrid,LF3dGrid}
    ∇2::SparseMatrixCSC{Float64,Int64}
    V_Ps_loc::Vector{Float64}
    V_Hartree::Vector{Float64}
    electrons::Electrons
    rhoe::Vector{Float64}
    atoms::Atoms
    precKin::Union{AMG_PREC_TYPE,ILU0Preconditioner}
    psolver::Union{PoissonSolverDAGE,PoissonSolverFFT}
    energies::Energies
    gvec::Union{Nothing,GVectors}
end
\end{juliacode}
\end{fullwidth}
%
A short explanation about the fields of \jlinline{Hamiltonian} follows.
\begin{itemize}
%
  \item \jlinline{grid}: the field that describes the real space grid points or
basis functions that is used to represent quantities such as wave functions,
potentials, and densities. In our case, this field is an instance of
\jlinline{FD3dGrid} or \jlinline{LF3dGrid}.
%
\item \jlinline{∇²}: the matrix representation of the ∇² operator
%
\item \jlinline{V_Ps_loc}: the local pseudopotential. It also will represent the
any external local potential that is felt by electrons.
Note that we have chosen the name \jlinline{V_Ps_loc} because of the
generalization that we will do in Chapter \ref{chap:ks_part_2}
%
\item \jlinline{V_Hartree}: the Hartree potential.
%
\item \jlinline{electrons}: an instance of type \jlinline{Electrons}. This field
stores various variables that are used to describe electron states such as number
of electrons, number of states, occupation numbers, etc.
%
\item \jlinline{rhoe}: the electron density
%
\item \jlinline{atoms}: an instance of \jlinline{Atoms}. This field can be used to represent
molecules, for example.
%
\item \jlinline{precKin}: the preconditioner based on the kinetic matrix. This preconditioner
can be for diagonalization of energy minimization.
%
\item \jlinline{psolver}: the Poisson equation solver which is used to calculate the Hartree
potential.
%
\item \jlinline{energies}: an instance of \jlinline{Energies}. This field stores
various total energy components.
%
\item \jlinline{gvec}: an instance of \jlinline{GVectors}. This field is only relevant for
periodic structure.
%
\end{itemize}


The following constructor can be used to initialize an instance of \jlinline{Hamiltonian}.
\begin{fullwidth}
\begin{juliacode}
function Hamiltonian( atoms::Atoms, grid, V_Ps_loc;
  Nelectrons=2, Nstates_extra=0, stencil_order=9,
  prec_type=:ILU0
)
  # Grid
  if typeof(grid) == FD3dGrid
      ∇2 = build_nabla2_matrix( grid, stencil_order=stencil_order )
  else
      ∇2 = build_nabla2_matrix( grid )
  end
  # Initialize gvec for periodic case
  if grid.pbc == (true,true,true)
      gvec = GVectors(grid)
  else
      gvec = nothing
  end
  Npoints = grid.Npoints
  V_Hartree = zeros(Float64, Npoints)
  Rhoe = zeros(Float64, Npoints)
  if prec_type == :amg
      precKin = aspreconditioner( ruge_stuben(-0.5*∇²) )
  else
      precKin = ILU0Preconditioner(-0.5*∇²)
  end
  electrons = Electrons( Nelectrons, Nstates_extra=Nstates_extra )
  if grid.pbc == (false,false,false)
      psolver = PoissonSolverDAGE(grid)
  else
      psolver = PoissonSolverFFT(grid)
  end
  energies = Energies()
  return Hamiltonian( grid, ∇², V_Ps_loc, V_Hartree, electrons,
                      Rhoe, atoms, precKin, psolver, energies, gvec )
end
\end{juliacode}
\end{fullwidth}

An important operation that must be defined for \jlinline{Hamiltonian} is the
multiplication between Hamiltonian and wave function.
This task is implemented in the function \jlinline{op_H}
\begin{juliacode}
function op_H( Ham::Hamiltonian, psi::Matrix{Float64} )
  Nbasis = size(psi,1)
  Nstates = size(psi,2)
  Hpsi = -0.5 * Ham.∇2 * psi
  for ist in 1:Nstates, ip in 1:Nbasis
      Hpsi[ip,ist] = Hpsi[ip,ist] + 
        ( Ham.V_Ps_loc[ip] + Ham.V_Hartree[ip] ) * psi[ip,ist]
  end
  return Hpsi
end
\end{juliacode}
In the function \jlinline{op_H}, first we apply the kinetic matrix by using the
\jlinline{∇²} field of \jlinline{Hamiltonian}, using sparse matrix
multiplication. This step is the followed by application of potential operator
which now consists of \jlinline{V_Ps_loc} and \jlinline{V_Hartree}.

We optionally can overload \jlinline{*} function so that we can use the usual
multiplication operator to carry out the action of Hamiltonian to wave function
\begin{juliacode}
import Base: *
function *( Ham::Hamiltonian, psi::Matrix{Float64} )
    return op_H(Ham, psi)
end
\end{juliacode}

We also define a function for updating the potentials (in this case only the Hartree potential, in
a full KS calculatio it also includes XC potential)
for an input electron density.
For our current purpose, this function do two things:
copy the input electron density to \jlinline{Ham.rhoe} and calculate the
Hartree potential by calling \jlinline{Poisson_solve} function.

\begin{juliacode}
function update!( Ham::Hamiltonian, Rhoe::Vector{Float64} )
    Ham.rhoe = Rhoe
    Ham.V_Hartree = Poisson_solve( Ham.psolver, Ham.grid, Rhoe )
    return
end
\end{juliacode}


\subsection{Electrons}

The type \jlinline{Electrons} stores several variables related to single-electron states.
\begin{juliacode}
mutable struct Electrons
    Nelectrons::Int64
    Nstates::Int64
    Nstates_occ::Int64
    Focc::Array{Float64,1}
    ene::Array{Float64,1}
end
\end{juliacode}

Electron density calculation \ref{eq:elec_dens_01}:
\begin{juliacode}
function calc_rhoe( Ham, psi::Array{Float64,2} )
    Nbasis = size(psi,1)
    Nstates = size(psi,2)
    Rhoe = zeros(Float64,Nbasis)
    for ist in 1:Nstates, ip in 1:Nbasis
        Rhoe[ip] = Rhoe[ip] +
          Ham.electrons.Focc[ist]*psi[ip,ist]*psi[ip,ist]
    end
    return Rhoe
end
\end{juliacode}


\subsection{Total energy terms}

We introduce the \jlinline{Energies} type to store various energy terms.
\begin{juliacode}
mutable struct Energies
    Kinetic::Float64
    Ps_loc::Float64
    Hartree::Float64
    NN::Float64
end
\end{juliacode}

Overload sum:
\begin{juliacode}
import Base: sum
function sum( ene::Energies )
    return ene.Kinetic + ene.Ps_loc + ene.Hartree + ene.NN
end
\end{juliacode}

Calculation of the kinetic energy:
\begin{equation}
E_{\mathrm{kin}} =
-\frac{1}{2} \int \psi_{i}(\mathbf{r}) \nabla^{2} \psi_{i}(\mathbf{r})\,\mathrm{d}\mathbf{r}
\end{equation}
can be done using the following function.
\begin{fullwidth}
\begin{juliacode}
function calc_E_kin( Ham::Hamiltonian, psi::Array{Float64,2} )
    Nbasis = size(psi,1)
    Nstates = size(psi,2)
    E_kin = 0.0
    nabla2psi = zeros(Float64,Nbasis)
    dVol = Ham.grid.dVol
    for ist in 1:Nstates
        @views nabla2psi = -0.5*Ham.∇2*psi[:,ist]
        @views E_kin = E_kin + Ham.electrons.Focc[ist]*dot(psi[:,ist],nabla2psi[:])*dVol
    end
    return E_kin
end
\end{juliacode}
\end{fullwidth}

In the function \jlinline{calc_energies!}, we calculate total electronic energy terms.
This function modifies \jlinline{Ham.energies}.
\begin{juliacode}
function calc_energies!( Ham::Hamiltonian, psi::Array{Float64,2} )
  dVol = Ham.grid.dVol
  Ham.energies.Kinetic = calc_E_kin( Ham, psi )
  Ham.energies.Ps_loc = sum( Ham.V_Ps_loc .* Ham.rhoe )*dVol
  Ham.energies.Hartree = 0.5*sum( Ham.V_Hartree .* Ham.rhoe )*dVol
  return
end
\end{juliacode}
It is sometimes convenient to return the energies directly. This is done by
the function \jlinline{calc_energies} which just wraps \jlinline{calc_energies!}
function.
\begin{juliacode}
function calc_energies(Ham, psi)
  calc_energies!(Ham, psi)
  return Ham.energies
end
\end{juliacode}


\section{Examples Hartree theory calculations}

In this section, we will begin our implementation of SCF Hatree theory for several
simple systems.
As in the previous chapters, we will start fromm a system with 3d harmonic potential as the
external potential. We will also 

\subsection{Harmonic potential}

First, we define the grid.
\begin{juliacode}
AA = [-3.0, -3.0, -3.0]
BB = [3.0, 3.0, 3.0]
NN = [25, 25, 25]
grid = FD3dGrid( NN, AA, BB )
\end{juliacode}

The, we calculate the external potential. We will use \jlinline{V_Ps_loc} as the name of
the external potential even though we are not dealing with pseudopotential.
We also choose $\omega=2$.
\begin{juliacode}
V_Ps_loc = pot_harmonic( grid, ω=2 )
\end{juliacode}

The next step is to initialize an instance of \jlinline{Hamiltonian}. We need to specify
number of electrons and number of states for our electronic states. Here we choose to 8
electrons and 4 states, each states is doubly occupied.
\begin{juliacode}
Nelectrons = 8
Nstates = round(Int64,Nelectrons/2)
Ham = Hamiltonian( Atoms(), grid, V_Ps_loc, Nelectrons=Nelectrons )
\end{juliacode}

After that, we prepare random wave functions for guess solution to the Kohn-Sham equation.
\begin{juliacode}
Npoints = grid.Npoints
dVol = grid.dVol
psi = rand(Float64,Npoints,Nstates)
ortho_sqrt!(psi)
psi = psi/sqrt(dVol)
\end{juliacode}
and calculate the electron density associated with these wave functions.
We also printed the integrated electron density, approximated with simple summation.
Note that the integrated electron density should be close the the number of electrons.
\begin{juliacode}
Rhoe = calc_rhoe( Ham, psi )
@printf("Integrated Rhoe = %18.10f\n", sum(Rhoe)*dVol)
\end{juliacode}

From the guess electron density, we update the Hamiltonian, calculate the
Hartree potential by calling the \jlinline{update!} function.
\begin{juliacode}
update!( Ham, Rhoe )
\end{juliacode}

We also calculate total energy for our initial guess wave functions and electron
density.
\begin{juliacode}
Etot = sum( calc_energies( Ham, psi ) )
@printf("Initial Etot = %18.10f\n", Etot)
\end{juliacode}

Before entering the SCF cycle, we need to prepare and define several variables
which will be used in the SCF cycle.
\begin{juliacode}
evals = zeros(Float64,Nstates)
Etot_old = Etot
dEtot = 0.0
dRhoe = 0.0
betamix = 0.5
NiterMax = 100
\end{juliacode}
An important variable that needs attention is \jlinline{betamix}. This variable
plays the role of $\beta$ in the following equation.
\begin{equation}
\rho^{i+1}_{\mathrm{in}}(\mathbf{r}) = \beta\rho(\mathbf{r})^{i}_{\mathrm{out}} +
(1 - \beta)\rho^{i}_{\mathrm{in}}(\mathbf{r})
\label{eq:linear_mix_rhoe}
\end{equation}
where $0 < \beta <= 1$.
In the Equation \eqref{eq:linear_mix_rhoe}, $\rho^{i+1}_{\mathrm{in}}(\mathbf{r})$
is the input density for the next SCF iteration.

The SCF cycle is implemented in the following Julia code.
\begin{juliacode}
for iterSCF in 1:NiterMax
  # Diagonalize the Hamiltonian
  evals = diag_LOBPCG!(Ham, psi, Ham.precKin)
  psi = psi/sqrt(dVol)
  # Calculate output electron density
  Rhoe_new = calc_rhoe(Ham, psi)
  # Mix electron density
  Rhoe = betamix*Rhoe_new + (1-betamix)*Rhoe
  # Update potential
  update!( Ham, Rhoe )
  # Calculate total energy
  Etot = sum( calc_energies( Ham, psi ) )
  dRhoe = sum(abs.(Rhoe - Rhoe_new))/Npoints
  dEtot = abs(Etot - Etot_old)
  @printf("%5d %18.10f %18.10e %18.10e\n", iterSCF, Etot, dEtot, dRhoe)
  # Check convergence
  if dEtot < 1e-6
    @printf("Convergence is achieved in %d iterations\n", iterSCF)
    for i in 1:Nstates
      @printf("%3d %18.10f\n", i, evals[i])
    end
    break
  end
  Etot_old = Etot
end
\end{juliacode}

The result of the SCF:
\begin{textcode}
... snipped
  33      62.2907110243   1.4097586885e-06   5.3155705598e-08
  34      62.2907100832   9.4107285520e-07   4.5940110326e-08
Convergence is achieved in 34 iterations
 1       9.8038626839
 2      11.1505190947
 3      11.1505191402
 4      11.1505192492
----------------------------
Total energy components
----------------------------
Kinetic =      12.9807771382
Ps_loc  =      25.0898019726
Hartree =      24.2201309724
NN      =       0.0000000000
----------------------------
Total   =      62.2907100832
\end{textcode}

Here is the result obtained using Octopus.
\begin{textcode}
 *********************** SCF CYCLE ITER #   18 ************************
 etot  =  6.22907491E+01 abs_ev   =  1.22E-04 rel_ev   =  1.41E-06
 ediff =       -1.13E-04 abs_dens =  6.72E-05 rel_dens =  8.39E-06
Matrix vector products:     27
Converged eigenvectors:      0

#  State  Eigenvalue [H]  Occupation    Error
      1        9.803876    2.000000   (9.3E-06)
      2       11.150521    2.000000   (9.3E-06)
      3       11.150524    2.000000   (7.1E-06)
      4       11.150525    2.000000   (9.6E-06) 
\end{textcode}

We get similar result with Octopus.



\subsection{H atom}

A next example that we will try is hydrogen atom. Here is our setup.
\begin{juliacode}
AA = [-8.0, -8.0, -8.0]
BB = [ 8.0,  8.0,  8.0]
NN = [41, 41, 41]
grid = FD3dGrid( NN, AA, BB )
atoms = Atoms( xyz_string=
  """
  1

  H  0.0  0.0  0.0
  """ )
V_Ps_loc = pot_Hps_HGH(atoms, grid)
Nstates = 1
Nelectrons = 1
Ham = Hamiltonian(atoms, grid, V_Ps_loc, Nelectrons=1)
\end{juliacode}

Result:
\begin{textcode}
... snipped
14      -0.2562610358   2.0600214368e-06   1.9590484379e-08
15      -0.2562602740   7.6184773828e-07   9.7155909123e-09
Convergence is achieved in 15 iterations
1      -0.0428298539
----------------------------
Total energy components
----------------------------
Kinetic =       0.2928408518
Ps_loc  =      -0.7625329979
Hartree =       0.2134318721
NN      =       0.0000000000
----------------------------
Total   =      -0.2562602740
\end{textcode}


{\center
\includegraphics[width=\textwidth]{../codes/hartree_scf/LOG_files/IMG_H.pdf}
\par}




\subsection{$\ce{H2}$ molecule}


Using the same grid as we have used for hydrogen atom.

\begin{juliacode}
atoms = Atoms( xyz_string=
  """
  2
  
  H   0.75  0.0  0.0
  H  -0.75  0.0  0.0
  """, in_bohr=true)
V_Ps_loc = pot_Hps_HGH(atoms, grid)
Nstates = 1
Nelectrons = 2
Ham = Hamiltonian(atoms, grid, V_Ps_loc, Nelectrons=Nelectrons)
\end{juliacode}

We need to calculate ion-ion interaction energy. Do this before SCF cycle.
\begin{juliacode}
Ham.energies.NN = calc_E_NN( [1.0, 1.0], atoms.positions )
\end{juliacode}

Result:
\begin{textcode}
... snipped
16      -0.5972202712   1.3097904592e-06   1.0969070600e-08
17      -0.5972197417   5.2951699026e-07   5.4868596190e-09
Convergence is achieved in 17 iterations
1      -0.1083633271
----------------------------
Total energy components
----------------------------
Kinetic =       0.8229371585
Ps_loc  =      -3.1339854624
Hartree =       1.0471618956
NN      =       0.6666666667
----------------------------
Total   =      -0.5972197417
\end{textcode}

\begin{figure}[h]
\begin{center}
\includegraphics[width=\textwidth]{../codes/hartree_scf/LOG_files/IMG_H2.pdf}
\end{center}
\caption{Caption XXX}
\end{figure}




\section{Kohn-Sham calculations}

We are now ready to include the XC term.

\begin{equation}
E_{\mathrm{xc}}[\rho(\mathbf{r})] = \int \rho(\mathbf{r})
\varepsilon_{xc}(\rho(\mathbf{r}))\,\mathrm{d}\mathbf{r}
\end{equation}

Decomposition:
\begin{equation}
E_{xc} = E_{x} + E_{c}
\end{equation}

\subsection{Exchange term: Slater exchange}

Homogeneous electron, exchange energy:
\begin{equation}
E_{x}[\rho(\mathbf{r})] = -\frac{3}{4} \left(\frac{3}{\pi}\right)^{1/3}
\int \rho(\mathbf{r})^{4/3}\,\mathrm{d}\mathbf{r}
\end{equation}

Or can be written as:
\begin{equation}
E_{x}[\rho(\mathbf{r})] = \int \rho(\mathbf{r}) \varepsilon_{x}(\rho(\mathbf{r}))\,\mathrm{d}\mathbf{r}
\end{equation}

Exchange energy density:
\begin{equation}
\varepsilon_{x}(\rho) = C_{x}\rho^{1/3}
\end{equation}
with parameter:
\begin{equation}
C_{x} = \frac{3}{4}\left(\frac{3}{\pi}\right)^{1/3}  
\end{equation}


\subsection{Correlation VWN}

VWN parameterization:
\begin{equation}
\varepsilon_{c} = A \left\{
\mathrm{ln}\frac{x}{X(x)} + \frac{2b}{Q}\mathrm{atan}\frac{Q}{2x+b}
- \frac{bx_{0}}{X(x_{0})}\left[
\mathrm{ln}\frac{(x - x_0)^2}{X(x)} + \frac{2(b + 2x_0)}{Q}\mathrm{atan}\frac{Q}{2x + b}
\right]
\right\}
\end{equation}

\begin{align}
x & = \sqrt{r_{s}} \\
r_s & = \left( \frac{3}{4\pi\rho} \right)^{1/3} \\
X(x) & = x^2 + bx + c \\
Q & = \sqrt{4c - b^2}
\end{align}

Parameter:
\begin{align}
A & = 0.0310907 \\
b & = 3.72744 \\
c & = 12.9352 \\
x_0 & = -0.10498
\end{align}

Wigner-Seitz radius is defined by:
\begin{equation}
\frac{4}{3}\pi r^{3}_{s} = \frac{1}{\rho}
\end{equation}

Potential:
\begin{equation}
V^{LDA}_{xc} = \frac{\delta E^{LDA}_{xc}}{\delta \rho(\mathbf{r})} = 
\varepsilon_{\mathrm{xc}}( \rho(\mathbf{r}) ) + \rho(\mathbf{r})
\frac{\partial \varepsilon_{\mathrm{xc}}( \rho(\mathbf{r}) )}{\partial\rho(\mathbf{r})}
\end{equation}


\subsection{Implementation}

New Hamiltonian, include \jlinline{V_XC}.

Update the potential:

\begin{juliacode}
function update!( Ham::Hamiltonian, Rhoe::Vector{Float64} )
  Ham.rhoe = Rhoe
  Ham.V_Hartree = Poisson_solve( Ham.psolver, Ham.grid, Rhoe )
  Ham.V_XC = calc_Vxc_VWN( XCCalculator(), Rhoe )
  return
end
\end{juliacode}

Application of Hamiltonian

\begin{juliacode}
function op_H( Ham::Hamiltonian, psi::Matrix{Float64} )
  Nbasis = size(psi,1)
  Nstates = size(psi,2)
  Hpsi = zeros(Float64,Nbasis,Nstates)
  Hpsi = -0.5*Ham.∇² * psi
  for ist in 1:Nstates, ip in 1:Nbasis
    Hpsi[ip,ist] = Hpsi[ip,ist] + ( Ham.V_Ps_loc[ip] + Ham.V_Hartree[ip] +
                   Ham.V_XC[ip] ) * psi[ip,ist]
  end
  return Hpsi
end
\end{juliacode}

Calculation of XC energy:
\begin{juliacode}
epsxc = calc_epsxc_VWN( XCCalculator(), Ham.rhoe )
Ham.energies.XC = sum( epsxc .* Ham.rhoe )*dVol
\end{juliacode}

\section{Examples of KS calculations}

\subsection{Harmonic potential}

Setup:
\begin{juliacode}
AA = [-3.0, -3.0, -3.0]
BB = [3.0, 3.0, 3.0]
NN = [25, 25, 25]
grid = FD3dGrid( NN, AA, BB )
V_Ps_loc = pot_harmonic( grid, ω=2 )
Nelectrons = 8
Nstates = round(Int64,Nelectrons/2)
Ham = Hamiltonian( Atoms(), grid, V_Ps_loc, Nelectrons=Nelectrons )
\end{juliacode}

\begin{textcode}
... snipped
  19      57.5543743808   1.0560459387e-06   4.0259990163e-05
  20      57.5543736562   7.2459251044e-07   3.1156747100e-05
Convergence is achieved in 20 iterations

Eigenvalues:
 1       9.0567277414
 2      10.4655966931
 3      10.4655970965
 4      10.4655977729
----------------------------
Total energy components
----------------------------
Kinetic =      13.6600637584
Ps_loc  =      23.8423941552
Hartree =      24.8463331194
XC      =      -4.7944173767
NN      =       0.0000000000
----------------------------
Total   =      57.5543736562
\end{textcode}


Result from Octopus:
\begin{fullwidth}
\begin{textcode}
*********************** SCF CYCLE ITER #   17 ************************
 etot  =  5.75543034E+01 abs_ev   =  4.97E-05 rel_ev   =  6.14E-07
 ediff =       -5.09E-05 abs_dens =  6.73E-05 rel_dens =  8.42E-06
Matrix vector products:     31
Converged eigenvectors:      0

#  State  Eigenvalue [H]  Occupation    Error
      1        9.056759    2.000000   (9.9E-06)
      2       10.465603    2.000000   (1.2E-05)
      3       10.465607    2.000000   (5.9E-06)
      4       10.465608    2.000000   (7.1E-06) 
\end{textcode}
\end{fullwidth}

Our result is similar to Octopus result.


\subsection{Hydrogen atom}

Result:
\begin{textcode}
  17      -0.4697222378   2.0241354490e-06   4.7656241007e-09
  18      -0.4697214473   7.9055452079e-07   2.4859509631e-09
Convergence is achieved in 18 iterations

Eigenvalues:
 1      -0.2449501107
----------------------------
Total energy components
----------------------------
Kinetic =       0.5027978737
Ps_loc  =      -1.0263363191
Hartree =       0.2995317605
XC      =      -0.2457147624
NN      =       0.0000000000
----------------------------
Total   =      -0.4697214473
\end{textcode}

Octopus result:
\begin{fullwidth}
\begin{textcode}
*********************** SCF CYCLE ITER #   12 ************************
 etot  = -4.70593023E-01 abs_ev   =  8.06E-07 rel_ev   =  3.28E-06
 ediff =       -4.65E-11 abs_dens =  9.21E-06 rel_dens =  9.21E-06
Matrix vector products:      6
Converged eigenvectors:      0
 
 #  State  Eigenvalue [H]  Occupation    Error
       1       -0.245334    1.000000   (3.1E-06)
\end{textcode}
\end{fullwidth}

Quite similar to Octopus.
Difference probabably due to the Poisson solver.

Need to study convergence with respect to grid spacing.

{\centering
\includegraphics[width=\textwidth]{../codes/ks_dft_02/LOG_files/IMG_H.pdf}
}

\subsection{$\ce{H2}$ molecule}

Result:
\begin{textcode}
  15      -1.1975841508   4.0237235348e-06   3.8822972895e-08
  16      -1.1975850469   8.9614081089e-07   2.1763673163e-08
Convergence is achieved in 16 iterations

Eigenvalues:
 1      -0.3827444348
----------------------------
Total energy components
----------------------------
Kinetic =       1.1958829137
Ps_loc  =      -3.7020236680
Hartree =       1.3006817627
XC      =      -0.6587927219
NN      =       0.6666666667
----------------------------
Total   =      -1.1975850469
\end{textcode}

Octopus result:
\begin{fullwidth}
\begin{textcode}
********************** SCF CYCLE ITER #   12 ************************
 etot  = -1.19927504E+00 abs_ev   =  1.50E-06 rel_ev   =  1.95E-06
 ediff =        1.68E-06 abs_dens =  2.11E-06 rel_dens =  1.06E-06
Matrix vector products:     15
Converged eigenvectors:      1

#  State  Eigenvalue [H]  Occupation    Error
      1       -0.383095    2.000000   (6.0E-07) 
\end{textcode}
\end{fullwidth}

Similar observation as H atom.

{\centering
\includegraphics[width=\textwidth]{../codes/ks_dft_02/LOG_files/IMG_H2.pdf}
}

\section{Exercise}

SCF convergence. Vary the \jlinline{betamix} parameter. Observe or the convergence
of SCF with respect to betamix.

Try adding one empty orbitals. Visualize the orbitals. Compare the results of
KS and Hartree. Are there any difference?

Calculate KS for periodic H-chain.
\chapter{Kohn-Sham equation part II}
\label{chap:ks_part_2}

In the previous chapters we have used pseudopotentials to model the interaction
between electrons and nuclei. The pseudopotentials that we have used
are limited to local form. In this case, the potential operator is diagonal
in real space. It turns out that it is very difficult to construct local pseudopotentials
that have good accuracy and transferability. Most of the pseudopotentials that
are used in density functional calculations have nonlocal components.

In this chapter, we will use a family of pseudopotentials that was
proposed by Goedecker-Teter-Hutter (GTH) \cite{Goedecker1996} in 1996.
We have used local-only component of this pseudopotential in the previous chapters.
Now, we will start to consider the nonlocal component.

The GTH pseudopotentials can be written in terms of
local $V^{\mathrm{PS}}_{\mathrm{loc}}$ and
angular momentum $l$ dependent
nonlocal components $\Delta V^{\mathrm{PS}}_{l}$:
\begin{equation}
V_{\mathrm{ele-nuc}}(\mathbf{r}) =
\sum_{I} \left[
V^{\mathrm{PS}}_{\mathrm{loc}}(\mathbf{r}-\mathbf{R}_{I}) +
\sum_{l=0}^{l_{\mathrm{max}}}
V^{\mathrm{PS}}_{l}(\mathbf{r}-\mathrm{R}_{I},\mathbf{r}'-\mathbf{R}_{I})
\right]
\end{equation}
%
The local pseudopotential for
$I$-th atom, $V^{\mathrm{PS}}_{\mathrm{loc}}(\mathbf{r}-\mathbf{R}_{I})$,
is radially symmetric
function with the following radial form
\begin{equation}
V^{\mathrm{PS}}_{\mathrm{loc}}(r) =
-\frac{Z_{\mathrm{val}}}{r}\mathrm{erf}\left[
\frac{\bar{r}}{\sqrt{2}} \right] +
\exp\left[-\frac{1}{2}\bar{r}^2\right]\left(
C_{1} + C_{2}\bar{r}^2 + C_{3}\bar{r}^4 + C_{4}\bar{r}^6
\right)
\label{eq:V_ps_loc_R}
\end{equation}
with $\bar{r}=r/r_{\mathrm{loc}}$ and $r_{\mathrm{loc}}$, $Z_{\mathrm{val}}$,
$C_{1}$, $C_{2}$, $C_{3}$ and $C_{4}$ are the corresponding pseudopotential
parameters.
In $\mathbf{G}$-space, the GTH local pseudopotential can be written as
\begin{multline}
V^{\mathrm{PS}}_{\mathrm{loc}}(G) = -\frac{4\pi}{\Omega}\frac{Z_{\mathrm{val}}}{G^2}
\exp\left[-\frac{x^2}{2}\right] +
\sqrt{8\pi^3} \frac{r^{3}_{\mathrm{loc}}}{\Omega}\exp\left[-\frac{x^2}{2}\right]\times\\
\left( C_{1} + C_{2}(3 - x^2) + C_{3}(15 - 10x^2 + x^4) + C_{4}(105 - 105x^2 + 21x^4 - x^6) \right)
\label{eq:V_ps_loc_G}
\end{multline}
where $x=G r_{\mathrm{loc}}$.
%
The nonlocal component of GTH pseudopotential can written in real space as
\begin{equation}
V^{\mathrm{PS}}_{l}(\mathbf{r}-\mathbf{R}_{I},\mathbf{r}'-\mathbf{R}_{I}) =
\sum_{\mu=1}^{N_{l}} \sum_{\nu=1}^{N_{l}} \sum_{m=-l}^{l}
\beta_{\mu lm}(\mathbf{r}-\mathbf{R}_{I})\,
h^{l}_{\mu\nu}\,
\beta^{*}_{\nu lm}(\mathbf{r}'-\mathbf{R}_{I})
\end{equation}
where $\beta_{\mu lm}(\mathbf{r})$ are atomic-centered projector functions
\begin{equation}
\beta_{\mu lm}(\mathbf{r}) = 
p^{l}_{\mu}(r) Y_{lm}(\hat{\mathbf{r}})
\label{eq:proj_NL_R}
\end{equation}
%
and $h^{l}_{\mu\nu}$ are the pseudopotential parameters and
$Y_{lm}$ are the spherical harmonics. Number of projectors per angular
momentum $N_{l}$ may take value up to 3 projectors.
The projectors can be written as
\begin{equation}
p^{l}_{\mu}(r) = \frac{\sqrt{2}}
{r^{l+(4i-1)/2}_{l}\sqrt{\Gamma(l + (4i-1)/2)}} r^{l+2(i-1)}
\exp\left[-\dfrac{r^2}{2r^{2}_{l}}\right] \, ,
\end{equation}
where $\Gamma(x)$ is the gamma function. The projectors are normalized according
to
\begin{equation}
\int_{0}^{\infty} r^2\,p^{l}_{i}(r)\,p^{l}_{i}(r)\,\mathrm{d}r = 1 \, .
\end{equation}

In the case of periodic sytem, the local part of the pseudopotential
is constructed using the formula in the $\mathbf{G}$-space 
and transformed them back to real space.
We refer the readers to the original
reference \cite{Goedecker1996} and the book \cite{Marx2009}
for more information about GTH pseudopotentials.

Due to the separation of local and non-local components of electrons-nuclei
interaction, interaction energy between electron and nuclei can be decomposed as
\begin{equation}
E_{\mathrm{ele-nuc}} = E^{\mathrm{PS}}_{\mathrm{loc}}
+ E^{\mathrm{PS}}_{\mathrm{nloc}}
\end{equation}
%
where the local pseudopotential contribution is
\begin{equation}
E^{\mathrm{PS}}_{\mathrm{loc}} =
\int_{\Omega} \rho(\mathbf{r})\,V^{\mathrm{PS}}_{\mathrm{loc}}(\mathbf{r})\,
\mathrm{d}\mathbf{r}
\end{equation}
%
and the non-local contribution is
\begin{equation}
E^{\mathrm{PS}}_{\mathrm{nloc}} = 
\sum_{i}
f_{i}
\int_{\Omega}\,
\psi^{*}_{i}(\mathbf{r})
\left[
\sum_{I}\sum_{l=0}^{l_{\mathrm{max}}}
V^{\mathrm{PS}}_{l}(\mathbf{r}-\mathbf{R}_{I},\mathbf{r}'-\mathbf{R}_{I})
\right]
\psi_{i}(\mathbf{r})
\,\mathrm{d}\mathbf{r}.
\end{equation}

\begin{juliacode}
struct PsPotNL
  NbetaNL::Int64
  prj2beta::Array{Int64,4}
  betaNL::Array{Float64,2}
end
\end{juliacode}

\begin{juliacode}
function PsPotNL()
  betaNL = zeros(Float64,1,1)
  return PsPotNL( 0, zeros(Int64,1,1,1,1), betaNL )
end
\end{juliacode}

\begin{juliacode}
function PsPotNL( atoms::Atoms, pspots::Array{PsPot_GTH,1}, grid; check_norm=false )
  Natoms = atoms.Natoms
  atm2species = atoms.atm2species
  atpos = atoms.positions

  prj2beta = Array{Int64}(undef,3,Natoms,4,7)
  prj2beta[:] .= -1   # set to invalid index

  NbetaNL = 0
  for ia = 1:Natoms
    isp = atm2species[ia]
    psp = pspots[isp]
    for l = 0:psp.lmax
      for iprj = 1:psp.Nproj_l[l+1]
        for m = -l:l
          NbetaNL = NbetaNL + 1
          prj2beta[iprj,ia,l+1,m+psp.lmax+1] = NbetaNL
        end
      end
    end
  end

  # No nonlocal components
  if NbetaNL == 0
    # return dummy PsPotNL
    betaNL = zeros(Float64,1,1)
    return PsPotNL( 0, zeros(Int64,1,1,1,1), betaNL )
  end

  Npoints = grid.Npoints
  betaNL = zeros(Float64, Npoints, NbetaNL)
  setup_betaNL!( atoms, grid, pspots, betaNL )

  return PsPotNL( NbetaNL, prj2beta, betaNL )
end
\end{juliacode}


The function setup:
\begin{juliacode}
function setup_betaNL!( atoms, grid, pspots, betaNL )
  Natoms = atoms.Natoms
  Npoints = grid.Npoints
  atm2species = atoms.atm2species

  ibeta = 0
  dr = zeros(3)
  for ia = 1:Natoms
    isp = atm2species[ia]
    psp = pspots[isp]
    for l = 0:psp.lmax
    for iprj = 1:psp.Nproj_l[l+1]
    for m = -l:l
      ibeta = ibeta + 1
      for ip in 1:Npoints
        dr[1] = grid.r[1,ip] - atoms.positions[1,ia]
        dr[2] = grid.r[2,ip] - atoms.positions[2,ia]
        dr[3] = grid.r[3,ip] - atoms.positions[3,ia]
        drm = sqrt( dr[1]^2 + dr[2]^2 + dr[3]^2 )
        betaNL[ip,ibeta] = Ylm_real(l, m, dr)*eval_proj_R(psp, l, iprj, drm)
      end
    end # m
    end # iprj
    end # l
  end
  return
end
\end{juliacode}

Calculate energy:
\begin{juliacode}
function calc_E_Ps_nloc( Ham::Hamiltonian, psi::Array{Float64,2} )
  # ... snipped, various shortcuts

  betaNL_psi = calc_betaNL_psi( Ham.pspotNL.betaNL, psi )*dVol

  E_Ps_nloc = 0.0
  for ist = 1:Nstates
    enl1 = 0.0
    for ia = 1:Natoms
      isp = atm2species[ia]
      psp = pspots[isp]
      for l = 0:psp.lmax, m = -l:l
        for iprj = 1:psp.Nproj_l[l+1], jprj = 1:psp.Nproj_l[l+1]
          ibeta = prj2beta[iprj,ia,l+1,m+psp.lmax+1]
          jbeta = prj2beta[jprj,ia,l+1,m+psp.lmax+1]
          hij = psp.h[l+1,iprj,jprj]
          enl1 = enl1 + hij * betaNL_psi[ist,ibeta] * betaNL_psi[ist,jbeta]
        end # jprj
      end # m, l
    end
    E_Ps_nloc = E_Ps_nloc + Focc[ist]*enl1
  end
  return E_Ps_nloc
end
\end{juliacode}

\begin{juliacode}
function op_V_Ps_nloc( Ham::Hamiltonian, psi::Array{Float64,2} )
  # ... snipped

  betaNL_psi = psi' * Ham.pspotNL.betaNL * dVol
  
  Vpsi = zeros(Float64,Npoints,Nstates)
  for ist = 1:Nstates
    for ia = 1:Natoms
      isp = atm2species[ia]
      psp = pspots[isp]
      for l = 0:psp.lmax, m = -l:l
        for iprj = 1:psp.Nproj_l[l+1], jprj = 1:psp.Nproj_l[l+1]
          ibeta = prj2beta[iprj,ia,l+1,m+psp.lmax+1]
          jbeta = prj2beta[jprj,ia,l+1,m+psp.lmax+1]
          hij = psp.h[l+1,iprj,jprj]
          for ip = 1:Npoints
              Vpsi[ip,ist] = Vpsi[ip,ist] + hij*betaNL[ip,ibeta]*betaNL_psi[ist,jbeta]
          end
        end # jprj, iprj
      end # m, l
    end # ia
  end # ist
  return Vpsi
end
\end{juliacode}


\begin{juliacode}
import Base: *
function *( Ham::Hamiltonian, psi::Matrix{Float64} )
  Nbasis = size(psi,1)
  Nstates = size(psi,2)
  Hpsi = -0.5*Ham.Laplacian*psi
  if Ham.pspotNL.NbetaNL > 0
    Vnlpsi = op_V_Ps_nloc(Ham, psi)
    for ist in 1:Nstates, ip in 1:Nbasis
        Hpsi[ip,ist] = Hpsi[ip,ist] + ( Ham.V_Ps_loc[ip] +
            Ham.V_Hartree[ip] + Ham.V_XC[ip] ) * psi[ip,ist] + Vnlpsi[ip,ist]
    end
  else # no nonlocal pspot components
    for ist in 1:Nstates, ip in 1:Nbasis
            Hpsi[ip,ist] = Hpsi[ip,ist] + ( Ham.V_Ps_loc[ip] +
                Ham.V_Hartree[ip] + Ham.V_XC[ip] ) * psi[ip,ist]
    end
  end
  return Hpsi
end
\end{juliacode}
\chapter{Epilogue}

Pointers to more advanced study materials, softwares.


\appendix
\chapter{Introduction to Julia programming language}

This chapter is intended to as an introduction to the Julia programming language.

This chapter assumes familiarity with command line interface.

\section{Installation}

Go to
{\scriptsize\url{https://julialang.org/downloads/}}
and download the suitable file for
your platform. For example, on 64 bit Linux OS, we can download the file
\txtinline{julia-1.x.x-linux-x86_64.tar.gz}
where \txtinline{1.x.x} referring to the version of Julia.
%
After you have downloaded the tarball you can unpack it.
%
\begin{bashcode}
tar xvf julia-1.x.x-linux-x86_64.tar.gz
\end{bashcode}
%
After unpacking the tarball, there should be a new folder called \txtinline{julia-1.x.x}.
You might want to put this directory under your home directory (or another directory of your
preference).

\section{Using Julia}

\subsection{Using Julia REPL}
Let's assume that you have put the Julia distribution under
your home directory. You can start the Julia interpreter by typing:
%
\begin{textcode}
/home/username/julia-1.x.x/bin/julia
\end{textcode}
%
You should see something like this in your terminal:
%
\begin{textcode}
$ julia 
               _
   _       _ _(_)_     |  Documentation: https://docs.julialang.org
  (_)     | (_) (_)    |
   _ _   _| |_  __ _   |  Type "?" for help, "]?" for Pkg help.
  | | | | | | |/ _` |  |
  | | |_| | | | (_| |  |  Version 1.1.1 (2019-05-16)
 _/ |\__'_|_|_|\__'_|  |  Official https://julialang.org/ release
|__/                   |

julia> 
\end{textcode}
%
This is called the Julia REPL (read-eval-print loop) or the Julia command prompt.
You can type the Julia program and see the output. This is useful for interactive
exploration or debugging the program.

The Julia code can be typed after the \txtinline{julia>} prompt. In this way,
we can write Julia code interactively.


Example Julia session
\begin{textcode}
julia> 1.2 + 3.4
4.6

julia> sin(2*pi)
-2.4492935982947064e-16

julia> sin(2*pi)^2 + cos(2*pi)^2
1.0
\end{textcode}

Using Unicode:
\begin{textcode}
julia> α = 1234;

julia> β = 3456;

julia> α * β
4264704
\end{textcode}

To exit type
\begin{textcode}
julia> exit()
\end{textcode}


\subsection{Julia script file}

In a text file with \txtinline{.jl} extension.

You can experiment with Julia REPL by typing \txtinline{julia}
at terminal:

We also can put the code in a text file with \txtinline{.jl} extension and
execute it with the command:
%
\begin{bashcode}
julia filename.jl
\end{bashcode}

The following code
\begin{juliacode}
function say_hello(name)
    println("Hello: ", name)
end
say_hello("efefer")
\end{juliacode}


\section{Basic programming construct}

Julia has similarities with several popular programming languages
such as Julia, MATLAB, and R, to name a few.


\section{Mathematical operators}

\begin{juliacode}
if a >= 1
  println("a is larger or equal to 1")
end
\end{juliacode}

Example code 3

\begin{juliacode}
using PGFPlotsX
using LaTeXStrings
include("init_FD1d_grid.jl")
function my_gaussian(x::Float64; α=1.0)
  return exp( -α*x^2 )
end
function main()
  A = -5.0
  B =  5.0
  Npoints = 8
  x, h = init_FD1d_grid( A, B, Npoints )

  NptsPlot = 200
  x_dense = range(A, stop=5, length=NptsPlot)

  f = @pgf(
    Axis( {height = "6cm", width = "10cm" },
      PlotInc( {mark="none"}, Coordinates(x_dense, my_gaussian.(x_dense)) ),
      LegendEntry(L"f(x)"),
      PlotInc( Coordinates(x, my_gaussian.(x)) ),
      LegendEntry(L"Sampled $f(x)$"),
    )
  )
  pgfsave("TEMP_gaussian_1d.pdf", f)
end
main()
\end{juliacode}

\chapter{Lagrange basis functions}

\section{Periodic Lagrange function}

For a given interval $[0,L]$, with $L>0$, the grid points $x_{i}$
appropriate for periodic Lagrange function are given by:

\begin{equation}
x_{i}=\frac{L}{2}\frac{2i-1}{N}
\end{equation}
with $i=1,\ldots,N$. Number of points $N$ should be an odd number.

The periodic cardinal functions $L_{i}^{\mathrm{per}}(x)$, defined
at grid point $i$ are given by:
\begin{equation}
L_{i}^{\mathrm{per}}(x)=\frac{1}{\sqrt{NL}}\sum_{n=1}^{N}\cos\left(\frac{\pi}{L}(2n-N-1)(x-x_{i})\right).
\end{equation}
The expansion of periodic function in terms of Lagrange functions:
\begin{equation}
f(x)=\sum_{i=1}^{N}c_{i}L_{i}^{\mathrm{per}}(x)
\end{equation}
with expansion coefficients $c_{i}=\sqrt{L/N}f(x_{i})$. When doing
variational calculation, the cofficients $c_{i}$ are the variational
parameters. The actual function values $f(x_{i}$) at grid points
$x_{i}$ is obtained by $f(x_{i})=\sqrt{N/L}c_{i}$. The prefactor
is sometimes abbreviated by $h=L/N$ and is also referred to as scaling
factor.

Consider periodic potential in one dimension:
\begin{equation}
V(x+L)=V(x).
\end{equation}
Floquet-Bloch theorem states that the wave function solution for periodic
potentials can be written in the form:
\begin{equation}
\psi_{k}(x)=e^{\imath kx}\phi_{k}(x)
\end{equation}
where function $\phi_{k}(x)$ and its first derivative $\phi_{k}'(x)$
have the same periodicity as $V(x)$ and $k$ is a constant called
the crystal momentum. Substituting this expression to Schrodinger
equation we obtain:
\begin{equation}
\left[-\frac{\hbar^{2}}{2m}\left(\frac{\mathrm{d}^{2}}{\mathrm{d}x^{2}}+2\imath k\frac{\mathrm{d}}{\mathrm{d}x}-k^{2}\right)+V(x)\right]\phi_{k}(x)=E\phi_{k}(k).
\end{equation}


An alternative way of enforcing periodicity of the wave function is
to require that:
\begin{equation}
\psi_{k}(x+L)=e^{\imath kL}\psi_{k}(x).
\end{equation}
This condition follows from:
\begin{eqnarray*}
\psi_{k}(x+L) & = & e^{\imath k(x+L)}\phi_{k}(x+L)\\
 & = & e^{\imath k(x+L)}\phi_{k}(x)\\
 & = & e^{\imath kL}e^{\imath kx}\phi_{k}(x)\\
 & = & e^{\imath kL}\psi_{k}(x)
\end{eqnarray*}


Using periodic cardinal the Schrodinger equation for periodic potential
can be written as:
\begin{equation}
\sum_{j=1}^{N}\left[-\frac{\hbar^{2}}{2m}\left(D_{jl}^{(2)}+2\imath kD_{jl}^{(1)}-k^{2}\delta_{jl}\right)+V(j)\delta_{jl}\right]\phi(j)=E\phi(l)
\end{equation}
with $l=1,\ldots,N$. $D_{jl}^{(1)}$ are matrix elements of the first
derivatives:
\begin{equation}
D_{jl}^{(1)}=\begin{cases}
0 & j=l\\
-\dfrac{2\pi}{L}(-1)^{j-l}\left(2\sin\dfrac{\pi(j-l)}{N}\right)^{-1} & j\neq l
\end{cases}
\end{equation}
and $D_{jl}^{(2)}$ are matrix elements of the second derivatives,
$N'=(N-1)/2$:
\begin{equation}
D_{jl}^{(2)}=\begin{cases}
-\left(\dfrac{2\pi}{L}\right)^{2}\dfrac{N'(N'+1)}{3} & j=l\\
-\left(\dfrac{2\pi}{L}\right)^{2}(-1)^{j-l}\dfrac{\cos\left(\pi(j-l)/N\right)}{2\sin^{2}\left[\pi(j-l)/N\right]} & j\neq l
\end{cases}
\end{equation}
Note that, $D_{jl}^{(1)}$ is not symmetric, but $D_{jl}^{(1)}=-D_{lj}^{(1)}$.
Meanwhile, the second derivative matrix $D_{jl}^{(2)}$ is symetric,
i.e. $D_{jl}^{(2)}=D_{lj}^{(2)}$. With the above expressions, first
and second derivative of periodic cardinals can be expressed as
\begin{eqnarray}
\frac{\mathrm{d}}{\mathrm{d}x}L_{i}^{\mathrm{per}}(x) & = & \sum_{j=1}^{N}D_{ji}^{(1)}L_{j}^{\mathrm{per}}(x)\\
\frac{\mathrm{d}^{2}}{\mathrm{d}x^{2}}L_{i}^{\mathrm{per}}(x) & = & \sum_{j=1}^{N}D_{ji}^{(2)}L_{j}^{\mathrm{per}}(x)
\end{eqnarray}


The previous approach also can be extended to periodic potential in 3D:
\[
V(\mathbf{r})=V(x,y,z)=V\left(x+L_{x},y+L_{y},z+L_{z}\right)
\]

Using periodic LF, Schrodinger equation can be casted into the following form:
\begin{equation}
\left[-\dfrac{\hbar^{2}}{2m}\left(\nabla^{2}+2\imath\mathbf{k}\cdot\nabla-\mathbf{k}^{2}\right)+V(\mathbf{r})\right]\phi_{\mathbf{k}}(\mathbf{r})=E\ \phi_{\mathbf{k}}(\mathbf{r})
\end{equation}


\section{Cluster Lagrange function}

For a given interval $[A,B]$, with $B>A$, the grid points $x_{i}$
appropriate for cluster Lagrange function are given by:
\[
x_{i}=A+\frac{B-A}{N+1}i
\]
where $i=1,\ldots,N$. Number of points $N$ can be either odd or
even number.

The cluster Lagrange functions $L_{i}^{\mathrm{clu}}(x)$, defined
at grid point $i$ are given by:
\begin{equation}
L_{i}^{\mathrm{clu}}(x)=\frac{2}{\sqrt{(N+1)(B-A)}}\sum_{n=1}^{N}\sin\left(k_{n}(x_{i}-A)\right)\sin\left(k_{n}(x-A)\right).
\end{equation}
where $k_{n}=\pi n/(B-A)$. The expansion of a function $f(x)$ in
terms of cluster Lagrange functions:
\begin{equation}
f(x)=\sum_{i=1}^{N}c_{i}L_{i}^{\mathrm{clu}}(x)
\end{equation}
with expansion coefficients $c_{i}=\sqrt{(B-A)/(N+1)}f(x_{i})$. When
doing variational calculation, the cofficients $c_{i}$ are the variational
parameters. The actual function values $f(x_{i}$) at grid points
$x_{i}$ is obtained by $f(x_{i})=\sqrt{(N+1)/(B-A)}c_{i}$.

Matrix elements $D_{jl}^{(2)}$ of the second derivatives for cluster
Lagrange functions are
\begin{equation}
D_{jl}^{(2)}=\begin{cases}
-\dfrac{1}{2}\left(\dfrac{\pi}{B-A}\right)^{2}\dfrac{2(N+1)^{2}+1}{3}-\dfrac{1}{\sin^{2}\left[\pi j/(N+1)\right]} & j=l\\
-\dfrac{1}{2}\left(\dfrac{\pi}{B-A}\right)^{2}(-1)^{j-l}\left[\dfrac{1}{\sin^{2}\left[\dfrac{\pi(j-l)}{2(N+1)}\right]}-\dfrac{1}{\sin^{2}\left[\dfrac{\pi(j+l)}{2(N+1)}\right]}\right] & j\neq l
\end{cases}
\end{equation}

For free or cluster boundary condition, we don't need $D_{jl}^{(1)}$.

\chapter{Introduction to \textsf{Octopus} package}

\textsf{Octopus} is a software package that can be used to perform electronic
structure calculations based on various methods including using density functional theory and
time-dependent density functional theory. It also uses finite-difference discretization
of spatial domain and can do calculations in 1d, 2d and 3d systems with various
boundary condition. In this book, \textsf{Octopus} is used to provide results which can
be compared with our own calculation.

\section{Installation}

Octopus is a command line program and is distributed in the source form or source code.
The source code should be compiled before use. The binary distribution might be available in
some platform OS or Linux distros.

The following packages are the mandatory requirement for compiling \textsf{Octopus}.
\begin{itemize}
\item Autotools and GNU Make
\item C, C++ and Fortran compilers
\item Libxc
\end{itemize}

Download the source tarball, extract it, change working directory to the extracted
directory and do the following in the terminal:
\begin{bashcode}
autoreconf --install
./configure --prefix=path_to_install
make
make install
\end{bashcode}



\section{A short description of input file}

Octopus input is written in an input file named \txtinline{inp}. A typical
input file might have the following content. Note that comments (starting with character
\#) are allowed in the input file.

\begin{textcode}
# Type of calculation: gs means ground state calculation
CalculationMode = gs

# Dimension of the system
Dimensions = 3

# Number of periodic dimension
PeriodicDimensions = 0

# Electronic structure theory that is used
TheoryLevel = dft
# XC correlation function, here we specify the VWN exchange correlation
XCFunctional = lda_x + lda_c_vwn_1

# Box shape
BoxShape = parallelepiped

# Lsize is half of box size
# Here we have have a box of size 16x16x16 bohr
%Lsize
 8 | 8 | 8
%

# Grid spacing (in bohr, unless otherwise specified)
spacing = 0.4

# Some self-consistent field settings
MixField = density
MixingScheme = Linear
Mixing = 0.5

# Pseudopotential set
PseudopotentialSet = hgh_lda

# Do not alter potential
FilterPotentials = filter_none

# Atomic coordinates
% Coordinates
  "H" | -0.75 | 0.0 | 0.0
  "H" |  0.75 | 0.0 | 0.0
%
\end{textcode}

In this case we are having isolated or non-periodic system.
For full 3d periodic dimension we can write:
\begin{textcode}
PeriodicDimensions = 3
\end{textcode}

For solving Schroedinger equation only (single particle):
\begin{textcode}
TheoryLevel = independent_particles
\end{textcode}

For Hartree-only calculation
\begin{textcode}
XCFunctional = none
\end{textcode}

Note that by default the potentials will be filtered in order to minimize egg-box effect.
The following is the default setting:
\begin{textcode}
FilterPotentials = filter_TS
\end{textcode}
We will usually use unfiltered pseudopotential as we generally do not implement any
schemes to reduce eggbox effect in our Julia code.

To execute the \textsf{Octopus} program and redirect the standard output and error to a
file named \txtinline{LOG_calc} we can type:
\begin{textcode}
octopus >LOG_calc 2>&1
\end{textcode}
After successful executation, several files are produced. For our purposes the
\txtinline{LOG_calc} file that contains standard output is sufficient. Usually we
are interested in the converged total energy and eigenvalues which can be found
near the end of \txtinline{LOG_calc} file.

Redirect stdout to a file (\txtinline{>out}), and then redirect stderr to stdout
(\txtinline{2>&1}).

Poisson equation:
\begin{textcode}
PoissonSolver = cg_corrected
PoissonSolverMaxMultipole = 4
\end{textcode}
Default value of isolated system: \txtinline{isf} (interpolating scaling function).

Harmonic potential (user defined potential):
\begin{textcode}
# Harmonic potential, with 8 valence electrons or 4 states
% Species 
  'HO' | species_user_defined | potential_formula | "2*(x^2 + y^2 + z^2)" | valence | 8
%

# Define center of the potential
% Coordinates
  "HO" | 0 | 0 | 0
%
\end{textcode}


Gaussian potential, using variable r and pi.
\begin{textcode}
% Species 
  "X" | species_user_defined | potential_formula | "-exp(-r^2)/sqrt(1/pi)^3" | valence | 2
%

% Coordinates
  "X" | 0 | 0 | 0
%
\end{textcode}

\bibliographystyle{unsrt}
\bibliography{BIBLIO}

\printindex

\end{document}
