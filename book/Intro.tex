\chapter{Introduction}

\section{Density functional theory}

This section is intended as a brief overview of Kohn-Sham density functional theory.
For more information, the reader is reffered to the original articles
\cite{Hohenberg1964,Kohn1965} and several books
\cite{Martin2004,Kohanoff2006,Marx2009,Giustino2014}.

Matters can be modeled as a collection of interacting electrons 
and nuclei. As such electrons shall be treated quantum mechanically  while
the nuclei can be considered as distributed points of positive charges. 
Surely, this 
collection forms a system with very complex internal interactions. 
Nevertheless the total energy of the system can simply be decomposed into 
kinetic and potential energies. The distributed nuclei of positive charges 
form Coulombic potential field affecting the interaction among the electrons 
in the system. Obviously, the electrons interaction will also affects the 
configuration of distributed nuclei. Deriving and solving an exact 
mathematical model of the system is an unforseeable future right now. 
Fortunately, by relaxing the exactness through some approximations this 
very complex system can be treated with ease and still provides very useful 
results. Since the mass of nuclei is about three order     
of magnitude heavier than the mass of electrons, the Born-Oppenheimer 
suggests an approximation that the nuclei are relatively fixed to fastly 
moving electrons. This approximation transforms the complex problems of 
interacting nuclei and electrons into problems of solely interacting 
many electrons in an external fields. For the interacting
electrons, the nuclei Coulombic potential field performs as external
potential. Nonetheless looking for the exact solution
wave functions for this many electrons problem is an impossible task. 
Wonderfully, with further approximations, computational or numerical 
methods provides the tractable ways to solve this extremely difficult 
problem.
Kohn-Sham density functional theory \cite{Hohenberg1964,Kohn1965}
is an attempt to offer useful  
model and computationally tractable solutions to this very complex 
system. The breakthrough provided by the Kohn-Sham approach is expressing 
the total energy as a functional of electrons density only. Marvelously 
the problem of looking for wave functions of many electrons problem 
becomes the much easier looking for electrons density.


In Kohn-Sham density functional theory, we are interested in total energy of a system of
interacting electrons
under external potential $V_{\mathrm{ext}}(\mathbf{r})$. This quantity can be written
as functional of
a set of single-particle wave functions or Kohn-Sham orbitals $\{\psi_{i}(\mathbf{r})\}$
\begin{equation}
E\left[\{\psi_{i}(\mathbf{r})\}\right] =
-\frac{1}{2} \int \psi_{i}(\mathbf{r}) \nabla^{2} \psi_{i}(\mathbf{r})\,\mathrm{d}\mathbf{r} +
\int \rho(\mathbf{r}) V_{\mathrm{ext}}(\mathbf{r})\,\mathrm{d}\mathbf{r} +
\frac{1}{2}\int
\frac{\rho(\mathbf{r}) \rho(\mathbf{r}')}{\left|\mathbf{r}-\mathbf{r}'\right|}\,
\mathrm{d}\mathbf{r}\,\mathrm{d}\mathbf{r}' + E_{\mathrm{xc}}\left[\rho(\mathbf{r})\right]
\label{eq:KS_energy_functional}
\end{equation}
%
where single-particle electron density $\rho(\mathbf{r})$ is calculated as
%
\begin{equation}
\rho(\mathbf{r}) = \sum_{i} f_{i} \psi^{*}(\mathbf{r}) \psi(\mathbf{r})
\label{eq:electron_density}
\end{equation}
%
In Equation \ref{eq:electron_density} the summation is done over
all occupied single electronic states $i$ and $f_{i}$
is the occupation number of the $i$-th orbital. For doubly-occupied orbitals
we have $f_{i}=2$.
In the usual setting in material science and chemistry, the external potential
is usually the potential due to the atomic nuclei (ions):
\begin{equation}
V_{\mathrm{ext}}(\mathbf{r}) = \sum_{I} \frac{Z_{I}}{\left|\mathbf{r} - \mathbf{R}_{I}\right|}
\end{equation}
and an additional term of energy, the nucleus-nucleus energy $E_{\mathrm{nn}}$,
is added to the total energy functional \eqref{eq:KS_energy_functional}:
\begin{equation}
E_{\mathrm{nn}} = \frac{1}{2} \sum_{I}\sum_{J}\frac{Z_{I} Z_{J}}{\left| R_{I} - R_{J} \right|}
\end{equation}

The energy terms in total energy functionals are the kinetic, external, Hartree, and exchange-correlation
(XC) energy, respectively. The functional form of the last term (i.e. the XC energy)
in terms of electron density is not known and one must resort to an approximation. In this article
we will use the local density approximation for the XC energy. Under this approximation, the XC energy
can be written as:
\begin{equation}
E_{\mathrm{xc}}\left[\rho(\mathbf{r})\right] =
\int \rho(\mathbf{r}) \epsilon_{\mathrm{xc}}\left[\rho(\mathbf{r})\right]\,\mathrm{d}\mathbf{r}
\end{equation}
where $\epsilon_{\mathrm{xc}}$ is the exchange-correlation energy per particle.
There are many functional forms that have been devised for $\epsilon_{\mathrm{xc}}$
and they are usually named by the persons who proposed them.
An explicit form of $\epsilon_{\mathrm{xc}}$ will be given later.


Many material properties can be derived from the minimum of the
functional \eqref{eq:KS_energy_functional}. This minimum energy is also called the
\textit{ground state energy}. This energy can be obtained by using direct
minimization of the Kohn-Sham energy functional or by solving the Kohn-Sham equations:
\begin{equation}
\hat{H}_{\mathrm{KS}}\,\psi_{i}(\mathbf{r}) = \epsilon_{i}\,\psi_{i}(\mathbf{r})
\label{eq:KS_equations}
\end{equation}
%
where $\epsilon_{i}$ are the Kohn-Sham orbital energies and the
Kohn-Sham Hamiltonian $\hat{H}_{\mathrm{KS}}$ is defined as
%
\begin{equation}
\hat{H}_{\mathrm{KS}} = -\frac{1}{2}\nabla^{2} + V_{\mathrm{ext}}(\mathbf{r}) +
V_{\mathrm{Ha}}(\mathbf{r}) + V_{\mathrm{xc}}(\mathbf{r})
\label{eq:KS_hamiltonian}
\end{equation}
%
From the solutions of the Kohn-Sham equations: ${\epsilon_{i}}$ and ${\psi_{i}(\mathbf{r})}$
we can calculate the corresponding minimum total energy from the functional
\eqref{eq:KS_energy_functional}.

In the definition of Kohn-Sham Hamiltonian, other than the external potential
which is usually specified from the problem,
we have two additional potential terms, namely the Hartree and exchange-correlation
potential.
The Hartree potential can be calculated from its integral form
\begin{equation}
V_{\mathrm{Ha}}(\mathbf{r}) = \int
\frac{\rho(\mathbf{r})}{\mathbf{r} - \mathbf{r}'}\,\mathrm{d}\mathbf{r}'
\end{equation}
An alternative way to calculate the Hartree potential is to solve the Poisson equation:
\begin{equation}
\nabla^2 V_{\mathrm{Ha}}(\mathbf{r}) = -4\pi\rho(\mathbf{r})
\label{eq:poisson_eq}
\end{equation}
%
The exchange-correlation potential is defined as functional derivative of the
exchange-correlation energy:
\begin{equation}
V_{\mathrm{xc}}(\mathbf{r}) = \frac{\delta E[\rho(\mathbf{r})]}{\delta \rho(\mathbf{r})}
\end{equation}

The Kohn-Sham equations are nonlinear eigenvalue equations in the sense that to calculate the
solutions $\{\epsilon_{i}\}$ and $\{\psi_{i}(\mathbf{r})\}$ we need to know
the electron density $\rho(\mathbf{r})$ to build the Hamiltonian. The electron density
itself must be calculated from $\{\psi_{i}(\mathbf{r})\}$ which are not known (the quantities
we want to solve for). The usual algorithm to solve the Kohn-Sham equations is the following:
\begin{itemize}
\item (\textbf{STEP 1}): start from a guess input density $\rho^{\mathrm{in}}(\mathbf{r})$
\item (\textbf{STEP 2}): calculate the Hamiltonian defined in \eqref{eq:KS_hamiltonian}
\item (\textbf{STEP 3}): solve the eigenvalue equations in \eqref{eq:KS_equations} to obtain the eigenpairs
  $\{\epsilon_{i}\}$, $\{\psi_{i}(\mathbf{r})\}$
\item (\textbf{STEP 4}): calculate the output electron density $\rho^{\mathrm{out}}(\mathbf{r})$
  from $\{\psi_{i}(\mathbf{r})\}$
  that are obtained from the previous step.
\item (\textbf{STEP 5}) Check whether the difference between $\rho^{\mathrm{in}}(\mathbf{r})$
  and $\rho^{\mathrm{out}}(\mathbf{r})$ is small.
  If the difference is still above a chosen threshold then back to \textbf{STEP 1}.
  If the difference is already small the stop the algorithm.
\end{itemize}
The algorithm we have just described is known as the self-consistent field (SCF) algorithm.

Nowadays, there are many available software packages that can be used to solve the Kohn-Sham
equations such as Quantum ESPRESSO, ABINIT, VASP, Gaussian, and NWChem are among the
popular ones. A more complete list is given in a Wikipedia page \cite{wiki-dft-softwares}.
These packages differs in several aspects, particularly the basis set used to discretize
the Kohn-Sham equations and characteristic of the systems they can handle (periodic or non-periodic
systems). These packages provide an easy way for researchers to carry out calculations based
on density functional theory for particular systems they are interested in without knowing the
details of how these calculations are performed.

\section{About this book}

In this book we will describe a simple way to solve the Kohn-Sham equations based on
finite difference approximation.
Our focus will be on the practical numerical implementation
of a solver for the Kohn-Sham equations.
The solver will be implemented using Julia programming language \cite{juliaorg}.
Our target is to calculate the ground state energy
of several simple model systems. We will begin to build our solver starting from the ground up.
The roadmap of the article is as follows.
\begin{itemize}
\item We begin from discussing numerical solution of Schroedinger equation in 1d.
  We will introduce finite difference approximation and its use in
  approximating second derivative operator that is present in the Schroedinger equation.
  We show how one can build the Hamiltonian matrix and
  solve the resulting eigenvalue equations using standard function that is available in Julia.
\item In the next section, we discuss numerical solution of Schroedinger in 2d. We will introduce
  how one can handle 2d grid and applying finite difference approximation to the Laplacian
  operator present in the 2d Schroedinger equation. We also present several iterative methods
  to solve the eigenvalue equations.
\item The next section discusses the numerical solution of Schroedinger equation in 3d. The methods
  presented in this section is a straightforward extension from the 2d case.
  In this section we start considering $V_{\mathrm{ext}}$ that is originated from the
  Coulomb interaction between atomic nuclei and electrons. We also introduce the concept of
  pseudopotential which is useful in practical calculations.
\item The next section discusses about Poisson equation. Conjugate gradient method for
  solving system of linear equations. Hartree energy calculation.
\item Hartree approximation, implementation of SCF algorithm
\item Kohn-Sham equations, XC energy and potential, hydrogen molecule
\end{itemize}

This book contains various program codes. They are written in Julia programming language
which is a dyanamic and interactive programming language with many similarities with
Python and MATLAB.
These codes are mostly in the form of snippets instead of full program. This means that
only important part to the code are included in the text. For a fully functioning code,
we refer the reader to the accompanying Julia programs for this book.