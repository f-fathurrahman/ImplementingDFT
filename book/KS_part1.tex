\chapter{Kohn-Sham equation part I}

Using local potential only

\section{Hartree calculation}
Ignoring the XC potential.

We will introduce self-consistent field (SCF) method.

New data structure: Hamiltonian

\begin{juliacode}
mutable struct Hamiltonian
    fdgrid::FD3dGrid
    Laplacian::SparseMatrixCSC{Float64,Int64}
    V_Ps_loc::Vector{Float64}
    V_Hartree::Vector{Float64}
    rhoe::Vector{Float64}
    precKin
    precLaplacian
end
\end{juliacode}

\begin{juliacode}
function Hamiltonian( fdgrid::FD3dGrid, ps_loc_func::Function; func_1d=build_D2_matrix_5pt )
    
    Laplacian = build_nabla2_matrix( fdgrid, func_1d=func_1d )
    
    V_Ps_loc = ps_loc_func( fdgrid )
    Npoints = fdgrid.Npoints
    
    V_Hartree = zeros(Float64, Npoints)

    Rhoe = zeros(Float64, Npoints)

    precKin = aspreconditioner( ruge_stuben(-0.5*Laplacian) )
    precLaplacian = aspreconditioner( ruge_stuben(Laplacian) )
    
    return Hamiltonian( fdgrid, Laplacian, V_Ps_loc, V_Hartree, Rhoe, precKin, precLaplacian )
end
\end{juliacode}


\begin{juliacode}
import Base: *
function *( Ham::Hamiltonian, psi::Matrix{Float64} )
    Nbasis = size(psi,1)
    Nstates = size(psi,2)
    Hpsi = zeros(Float64,Nbasis,Nstates)
    # include occupation number factor
    Hpsi = -0.5*Ham.Laplacian * psi #* 2.0
    for ist in 1:Nstates, ip in 1:Nbasis
        Hpsi[ip,ist] = Hpsi[ip,ist] + ( Ham.V_Ps_loc[ip] + Ham.V_Hartree[ip] ) * psi[ip,ist]
    end
    return Hpsi
end
\end{juliacode}

\begin{juliacode}
function update!( Ham::Hamiltonian, Rhoe::Vector{Float64} )
    Ham.rhoe[:] = Rhoe[:]
    Ham.V_Hartree = Poisson_solve_PCG( Ham.Laplacian, Ham.precLaplacian, -4*pi*Rhoe, 1000, verbose=false, TOL=1e-10 )
    return
end
\end{juliacode}

Calculate electron density:
\begin{juliacode}
function calc_rhoe( psi::Array{Float64,2} )
    Nbasis = size(psi,1)
    Nstates = size(psi,2)
    Rhoe = zeros(Float64,Nbasis)
    for ist in 1:Nstates
        for ip in 1:Nbasis
            Rhoe[ip] = Rhoe[ip] + 2.0*psi[ip,ist]*psi[ip,ist]
        end
    end
    return Rhoe
end
\end{juliacode}

Calculate energy terms:
\begin{juliacode}
mutable struct Energies
    Kinetic::Float64
    Ps_loc::Float64
    Hartree::Float64
end

import Base: sum
function sum( ene::Energies )
    return ene.Kinetic + ene.Ps_loc + ene.Hartree
end

function calc_E_kin( Ham, psi::Array{Float64,2} )
    Nbasis = size(psi,1)
    Nstates = size(psi,2)
    E_kin = 0.0
    nabla2psi = zeros(Float64,Nbasis)
    dVol = Ham.fdgrid.dVol
    # Assumption: Focc = 2 for all states
    for ist in 1:Nstates
        @views nabla2psi = -0.5*Ham.Laplacian*psi[:,ist]
        E_kin = E_kin + 2.0*dot( psi[:,ist], nabla2psi[:] )*dVol
    end
    return E_kin
end

function calc_energies( Ham::Hamiltonian, psi::Array{Float64,2} )
    dVol = Ham.fdgrid.dVol
    E_kin = calc_E_kin( Ham, psi )
    E_Ps_loc = sum( Ham.V_Ps_loc .* Ham.rhoe )*dVol
    E_Hartree = 0.5*sum( Ham.V_Hartree .* Ham.rhoe )*dVol

    return Energies(E_kin, E_Ps_loc, E_Hartree)
end
\end{juliacode}

Self-consisten field:
\begin{juliacode}
function pot_harmonic( fdgrid::FD3dGrid; ω=1.0, center=[0.0, 0.0, 0.0] )
    Npoints = fdgrid.Npoints
    Vpot = zeros(Npoints)
    for i in 1:Npoints
        x = fdgrid.r[1,i] - center[1]
        y = fdgrid.r[2,i] - center[2]
        z = fdgrid.r[3,i] - center[3]
        Vpot[i] = 0.5 * ω^2 *( x^2 + y^2 + z^2 )
    end
    return Vpot
end

function main()
    AA = [-3.0, -3.0, -3.0]
    BB = [3.0, 3.0, 3.0]
    NN = [25, 25, 25]

    fdgrid = FD3dGrid( NN, AA, BB )
    my_pot_harmonic( fdgrid ) = pot_harmonic( fdgrid, ω=2 )
    Ham = Hamiltonian( fdgrid, my_pot_harmonic, func_1d=build_D2_matrix_9pt )

    Nbasis = prod(NN)
    dVol = fdgrid.dVol
    Nstates = 4
    psi = rand(Float64,Nbasis,Nstates)
    ortho_sqrt!(psi)
    psi = psi/sqrt(dVol)

    Rhoe = calc_rhoe( psi )
    @printf("Integrated Rhoe = %18.10f\n", sum(Rhoe)*dVol)
    update!( Ham, Rhoe )

    evals = zeros(Float64,Nstates)
    Etot_old = 0.0
    dEtot = 0.0
    betamix = 0.5
    dRhoe = 0.0
    NiterMax = 100

    for iterSCF in 1:NiterMax
        evals = diag_LOBPCG!( Ham, psi, Ham.precKin, verbose_last=true )
        psi = psi/sqrt(dVol)
        Rhoe_new = calc_rhoe( psi )
        @printf("Integrated Rhoe_new = %18.10f\n", sum(Rhoe_new)*dVol)
        Rhoe = betamix*Rhoe_new + (1-betamix)*Rhoe
        @printf("Integrated Rhoe     = %18.10f\n", sum(Rhoe)*dVol)
        update!( Ham, Rhoe )
        Etot = sum( calc_energies( Ham, psi ) )
        dRhoe = norm(Rhoe - Rhoe_new)
        dEtot = abs(Etot - Etot_old)
        @printf("%5d %18.10f %18.10e %18.10e\n", iterSCF, Etot, dEtot, dRhoe)
        if dEtot < 1e-6
            @printf("Convergence is achieved in %d iterations\n", iterSCF)
            for i in 1:Nstates
                @printf("%3d %18.10f\n", i, evals[i])
            end
            break
        end

        Etot_old = Etot
    end
end
\end{juliacode}

