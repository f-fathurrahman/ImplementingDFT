\chapter{Kohn-Sham equation part I}

In this chapter we will put together tools that we have built in the previous chapters.
Our task is to solve the Kohn-Sham equation:
\begin{equation}
\left[ -\frac{1}{2}\nabla^2 + V_{\mathrm{KS}}(\mathbf{r}) \right]
\psi_{i}(\mathbf{r}) = \epsilon_{i} \psi_{i}(\mathbf{r})
\end{equation}
where $V_{\mathrm{KS}}(\mathbf{r})$ is effective single
particle potential or
the Kohn-Sham potential:
\begin{equation}
V_{\mathrm{KS}}(\mathbf{r}) =
V_{\mathrm{ext}}(\mathbf{r}) + V_{\mathrm{Ha}}(\mathbf{r}) + V_{\mathrm{xc}}(\mathbf{r})
\label{eq:KS_pot_local}
\end{equation}

The Kohn-Sham equation looks very much like Schroedinger equation, but with two additional
potentials: the Hartree and XC potential.
We will consider a 3d systems, so the techniques
that we have learned in Chapter \ref{chap:sch_3d} will be used for solving the
Kohn-Sham equation. Due to the presence of Hartree and XC potential, there will be additional
steps that need to be performed in order to solve the Kohn-Sham equation.
To calculate Hartree and XC potential, we need to calculate electron density which depend
on the solution of the Kohn-Sham equation itself. Because of this chicken-and-egg
characteristic, the Kohn-Sham equation must be solved self-consistently.

Solutions to the Kohn-Sham equation are the Kohn-Sham orbitals
$\psi_{i}(\mathrm{r})$ and eigenvalues $\epsilon_{i}$.
Total Kohn-Sham energy can be written as:
\begin{equation}
E^{\mathrm{KS}}_{\mathrm{tot}} = E_{\mathrm{kin}} + E_{\mathrm{ext}}
+ E_{\mathrm{Ha}} + E_{\mathrm{xc}}
\end{equation}
%
The kinetic energy of noninteracting electrons:
\begin{equation}
E_{\mathrm{kin}} = -\frac{1}{2}\sum_{i} \int \psi_{i}^{*}(\mathbf{r}) \nabla^2 \psi_{i}(\mathbf{r})
\,\mathrm{d}\mathbf{r}
\end{equation}
%
External potential energy:
\begin{equation}
E_{\mathrm{ext}} = \int \rho(\mathbf{r}) V_{\mathrm{ext}}(\mathbf{r})
\,\mathrm{d}\mathbf{r}
\end{equation}
%
Hartree energy:
\begin{equation}
E_{\mathrm{ext}} = \int \rho(\mathbf{r}) V_{\mathrm{Ha}}(\mathbf{r})
\,\mathrm{d}\mathbf{r}
\end{equation}
%
XC energy (LDA):
\begin{equation}
E_{\mathrm{ext}} = \int \varepsilon_{\mathrm{xc}}[\rho(\mathbf{r})] \rho(\mathbf{r})
\,\mathrm{d}\mathbf{r}
\end{equation}
where $\varepsilon_{\mathrm{xc}}$ is the XC energy per particle per volume.

Alternative expression of total energy as sum of orbital energies:
\begin{equation}
E^{\mathrm{KS}}_{\mathrm{tot}} = \sum_{i}^{N} \epsilon_{i} - E_{\mathrm{Ha}} + E_{\mathrm{xc}}
- \int \frac{\delta E_{\mathrm{xc}}}{\delta \rho(\mathbf{r})} \rho(\mathbf{r})
\,\mathrm{d}\mathbf{r}
\end{equation}

Nucleus-nucleus interaction energy:
\begin{equation}
E_{\mathrm{tot}} = E^{\mathrm{KS}}_{\mathrm{tot}} + E_{\mathrm{NN}}
\end{equation}

We first consider that the case of $V_{\mathrm{xc}}=0$. This will make our first implementation
of self-consistent field to be easier.

\section{Hartree calculation}

Hartree potential is the classical electrostatic interaction potential.
It is defined as
\begin{equation}
V_{\mathrm{Ha}}(\mathbf{r}) = \int \frac{\rho(\mathbf{r}')}{\left| \mathbf{r} - \mathbf{r}' \right|}
\,\mathrm{d}\mathbf{r}'
\end{equation}
Given the electron density:
\begin{equation}
\rho(\mathbf{r}) = \sum_{i} f_{i} \psi_{i}^{*}(\mathbf{r}) \psi_{i}(\mathbf{r})
\label{eq:elec_dens_01}
\end{equation}
we can use any methods in Chapter \ref{chap:poisson_3d} to calculate
$V_{\mathrm{Ha}}(\mathbf{r})$. The factor $f_{i}$ in the Equation \ref{eq:elec_dens_01}
is the occupation number of the electron.

\subsection{The Hamiltonian struct}

Because our Hamiltonian contain additional potential, its action to a wave function is
more complicated. We also need to implement several operations such as calculation
of electron density for given wave functions and total energy.
These calculations usually need access to several variables. To make access to various
variables easy, we will wrap them in a "big" Julia struct. We will call this struct
as \jlinline{Hamiltonian}.

The definition of the \jlinline{Hamiltonian} struct can be seen in the following
Julia code.
\begin{juliacode}
mutable struct Hamiltonian
  grid::Union{FD3dGrid,LF3dGrid}
  Laplacian::SparseMatrixCSC{Float64,Int64}
  V_Ps_loc::Vector{Float64}
  V_Hartree::Vector{Float64}
  electrons::Electrons
  rhoe::Vector{Float64}
  atoms::Atoms
  precKin
  psolver::Union{PoissonSolverDAGE,PoissonSolverFFT}
  energies::Energies
  gvec::Union{Nothing,GVectors}
end
\end{juliacode}

A short explanation about the fields of \jlinline{Hamiltonian} follows.
\begin{itemize}
%
  \item \jlinline{grid}: the field that describes the real space grid points or
basis functions that is used to represent quantities such as wave functions,
potentials, and densities. In our case, this field is an instance of
\jlinline{FD3dGrid} or \jlinline{LF3dGrid}.
%
\item \jlinline{Laplacian}: the matrix representation of the Laplacian operator
%
\item \jlinline{V_Ps_loc}: the local pseudopotential. It also will represent the
any external local potential that is felt by electrons.
Note that we have chosen the name \jlinline{V_Ps_loc} because of the
generalization that we will do in Chapter \ref{chap:ks_part_2}
%
\item \jlinline{V_Hartree}: the Hartree potential.
%
\item \jlinline{electrons}: an instance of type \jlinline{Electrons}. This field
stores various variables that are used to describe electron states such as number
of electrons, number of states, occupation numbers, etc.
%
\item \jlinline{rhoe}: the electron density
%
\item \jlinline{atoms}: an instance of \jlinline{Atoms}. This field can be used to represent
molecules, for example.
%
\item \jlinline{precKin}: the preconditioner based on the kinetic matrix. This preconditioner
can be for diagonalization of energy minimization.
%
\item \jlinline{psolver}: the Poisson equation solver which is used to calculate the Hartree
potential.
%
\item \jlinline{energies}: an instance of \jlinline{Energies}. This field stores
various total energy components.
%
\item \jlinline{gvec}: an instance of \jlinline{GVectors}. This field is only relevant for
periodic structure.
%
\end{itemize}


The following constructor can be used to initialize an instance of \jlinline{Hamiltonian}.

\begin{juliacode}
function Hamiltonian(
  atoms::Atoms, grid, V_Ps_loc;
  Nelectrons=2, Nstates_extra=0
)
  Laplacian = build_nabla2_matrix( grid )
  if grid.pbc == (true,true,true)
      gvec = GVectors(grid)
  else
      gvec = nothing
  end
  Npoints = grid.Npoints
  V_Hartree = zeros(Float64, Npoints)
  Rhoe = zeros(Float64, Npoints)
  precKin = aspreconditioner( ruge_stuben(-0.5*Laplacian) )
  electrons = Electrons( Nelectrons, Nstates_extra=Nstates_extra )
  if grid.pbc == (false,false,false)
      psolver = PoissonSolverDAGE(grid)
  else
      psolver = PoissonSolverFFT(grid)
  end
  energies = Energies()
  return Hamiltonian( grid, Laplacian, V_Ps_loc, V_Hartree, electrons,
                      Rhoe, atoms, precKin, psolver, energies, gvec )
end
\end{juliacode}

An important operation that must be defined for \jlinline{Hamiltonian} is the
multiplication between Hamiltonian and wave function.
We can define this operator by overloading the \jlinline{*} operator.
In the following code, first we apply the kinetic matrix by using the
\jlinline{Laplacian} field of \jlinline{Hamiltonian}, using sparse matrix
multiplication. This step is the followed by application of potential operator
which now consists of \jlinline{V_Ps_loc} and \jlinline{V_Hartree}.
\begin{juliacode}
import Base: *
function *( Ham::Hamiltonian, psi::Matrix{Float64} )
  Nbasis = size(psi,1)
  Nstates = size(psi,2)
  Hpsi = zeros(Float64,Nbasis,Nstates)
  Hpsi = -0.5 * Ham.Laplacian * psi
  for ist in 1:Nstates, ip in 1:Nbasis
    Hpsi[ip,ist] = Hpsi[ip,ist] + ( Ham.V_Ps_loc[ip] + Ham.V_Hartree[ip] ) * psi[ip,ist]
  end
  return Hpsi
end
\end{juliacode}


We also define a function for updating the potential for an input electron density.
For our current purpose, this function do two things:
copy the input electron density to \jlinline{Ham.rhoe} and calculate the
Hartree potential by calling \jlinline{Poisson_solve} function.

\begin{juliacode}
function update!( Ham::Hamiltonian, Rhoe::Vector{Float64} )
  Ham.rhoe = Rhoe
  Ham.V_Hartree = Poisson_solve( Ham.psolver, Ham.grid, Rhoe )
  return
end
\end{juliacode}


\subsection{Electrons}

The type \jlinline{Electrons} stores several variables related to single-electron states.
\begin{juliacode}
mutable struct Electrons
  Nelectrons::Int64
  Nstates::Int64
  Nstates_occ::Int64
  Focc::Array{Float64,1}
  ene::Array{Float64,1}
end
\end{juliacode}

Electron density calculation \ref{eq:elec_dens_01}:

\begin{juliacode}
function calc_rhoe( Ham, psi::Array{Float64,2} )
  Nbasis = size(psi,1)
  Nstates = size(psi,2)
  Rhoe = zeros(Float64,Nbasis)
  for ist in 1:Nstates
    f = Ham.electrons.Focc[ist]
    for ip in 1:Nbasis
      Rhoe[ip] = Rhoe[ip] + f*psi[ip,ist]*psi[ip,ist]
    end
  end
  return Rhoe
end
\end{juliacode}


\subsection{Total energy terms}

We introduce the \jlinline{Energies} type to store various energy terms.
\begin{juliacode}
mutable struct Energies
  Kinetic::Float64
  Ps_loc::Float64
  Hartree::Float64
  NN::Float64
end
\end{juliacode}

Overload sum:
\begin{juliacode}
import Base: sum
function sum( ene::Energies )
  return ene.Kinetic + ene.Ps_loc + ene.Hartree + ene.NN
end
\end{juliacode}

Calculation of the kinetic energy:
\begin{equation}
E_{\mathrm{kin}} =
-\frac{1}{2} \int \psi_{i}(\mathbf{r}) \nabla^{2} \psi_{i}(\mathbf{r})\,\mathrm{d}\mathbf{r}
\end{equation}
can be done using the following function.
\begin{juliacode}
function calc_E_kin( Ham::Hamiltonian, psi::Array{Float64,2} )
  Nbasis = size(psi,1)
  Nstates = size(psi,2)
  E_kin = 0.0
  nabla2psi = zeros(Float64,Nbasis)
  dVol = Ham.grid.dVol
  for ist in 1:Nstates
    @views nabla2psi = -0.5*Ham.Laplacian*psi[:,ist]
    @views E_kin = E_kin + Ham.electrons.Focc[ist]*dot( psi[:,ist], nabla2psi[:] )*dVol
  end
  return E_kin
end
\end{juliacode}

In the function \jlinline{calc_energies!}, we calculate total electronic energy terms.
This function modifies \jlinline{Ham.energies}.
\begin{juliacode}
function calc_energies!( Ham::Hamiltonian, psi::Array{Float64,2} )
  dVol = Ham.grid.dVol
  Ham.energies.Kinetic = calc_E_kin( Ham, psi )
  Ham.energies.Ps_loc = sum( Ham.V_Ps_loc .* Ham.rhoe )*dVol
  Ham.energies.Hartree = 0.5*sum( Ham.V_Hartree .* Ham.rhoe )*dVol
  return
end
\end{juliacode}
It is sometimes convenient to return the energies directly. This is done by
the function \jlinline{calc_energies} which just wraps \jlinline{calc_energies!}
function.
\begin{juliacode}
function calc_energies(Ham, psi)
  calc_energies!(Ham, psi)
  return Ham.energies
end
\end{juliacode}


\subsection{Harmonic potential}

We are now ready to implement our self-consistent field solution to the Kohn-Sham equation.
As in the previous chapters, we will chose a system with 3d harmonic potential as the
external potential.

First, we define the grid.
\begin{juliacode}
AA = [-3.0, -3.0, -3.0]
BB = [3.0, 3.0, 3.0]
NN = [25, 25, 25]
grid = FD3dGrid( NN, AA, BB )
\end{juliacode}

The, we calculate the external potential. We will use \jlinline{V_Ps_loc} as the name of
the external potential. We also choose $\omega=2$.
\begin{juliacode}
V_Ps_loc = pot_harmonic( grid, ω=2 )
\end{juliacode}

The next step is to initialize an instance of \jlinline{Hamiltonian}. We need to specify
number of electrons and number of states for our electronic states. Here we choose to 8
electrons and 4 states, each states is doubly occupied.
\begin{juliacode}
Nelectrons = 8
Nstates = round(Int64,Nelectrons/2)
Ham = Hamiltonian( Atoms(), grid, V_Ps_loc, Nelectrons=Nelectrons )
\end{juliacode}

We prepare random wave functions for guess solution to the Kohn-Sham equation.
\begin{juliacode}
Npoints = grid.Npoints
dVol = grid.dVol
psi = rand(Float64,Npoints,Nstates)
ortho_sqrt!(psi)
psi = psi/sqrt(dVol)
\end{juliacode}
and calculate the electron density associated with these wave functions.
We also printed the integrated electron density, approximated with simple summation.
Note that the integrated electron density should be close the the number of electrons.
\begin{juliacode}
Rhoe = calc_rhoe( Ham, psi )
@printf("Integrated Rhoe = %18.10f\n", sum(Rhoe)*dVol)
\end{juliacode}

From the guess electron density, we update the Hamiltonian, calculate the
Hartree potential by calling the \jlinline{update!} function.
\begin{juliacode}
update!( Ham, Rhoe )
\end{juliacode}

We also calculate total energy for our initial guess wave functions and electron
density.
\begin{juliacode}
Etot = sum( calc_energies( Ham, psi ) )
@printf("Initial Etot = %18.10f\n", Etot)
\end{juliacode}

Before entering the SCF cycle, we need to prepare and define several variables
which will be used in the SCF cycle.
\begin{juliacode}
evals = zeros(Float64,Nstates)
Etot_old = Etot
dEtot = 0.0
dRhoe = 0.0
betamix = 0.5
NiterMax = 100
\end{juliacode}
An important variable that needs attention is \jlinline{betamix}. This variable
plays the role of $\beta$ in the following equation.
\begin{equation}
\rho^{i+1}_{\mathrm{in}}(\mathbf{r}) = \beta\rho(\mathbf{r})^{i}_{\mathrm{out}} +
(1 - \beta)\rho^{i}_{\mathrm{in}}(\mathbf{r})
\label{eq:linear_mix_rhoe}
\end{equation}
where $0 < \beta <= 1$.
In the Equation \eqref{eq:linear_mix_rhoe}, $\rho^{i+1}_{\mathrm{in}}(\mathbf{r})$
is the input density for the next SCF iteration.

The SCF cycle is implemented in the following Julia code.
\begin{juliacode}
for iterSCF in 1:NiterMax
  evals = diag_LOBPCG!(Ham, psi, Ham.precKin)
  psi = psi/sqrt(dVol)
  Rhoe_new = calc_rhoe(Ham, psi)
  Rhoe = betamix*Rhoe_new + (1-betamix)*Rhoe
  update!( Ham, Rhoe )
  Etot = sum( calc_energies( Ham, psi ) )
  dRhoe = sum(abs.(Rhoe - Rhoe_new))/Npoints
  dEtot = abs(Etot - Etot_old)
  @printf("%5d %18.10f %18.10e %18.10e\n", iterSCF, Etot, dEtot, dRhoe)
  if dEtot < 1e-6
    @printf("Convergence is achieved in %d iterations\n", iterSCF)
    for i in 1:Nstates
      @printf("%3d %18.10f\n", i, evals[i])
    end
    break
  end
  Etot_old = Etot
end
\end{juliacode}

The result of the SCF:
\begin{textcode}
... snipped
  33      62.2907110243   1.4097586885e-06   5.3155705598e-08
  34      62.2907100832   9.4107285520e-07   4.5940110326e-08
Convergence is achieved in 34 iterations
 1       9.8038626839
 2      11.1505190947
 3      11.1505191402
 4      11.1505192492
----------------------------
Total energy components
----------------------------
Kinetic =      12.9807771382
Ps_loc  =      25.0898019726
Hartree =      24.2201309724
NN      =       0.0000000000
----------------------------
Total   =      62.2907100832
\end{textcode}




\subsection{H atom}

A next example that we will try is hydrogen atom. Here is our setup.
\begin{juliacode}
AA = [-8.0, -8.0, -8.0]
BB = [ 8.0,  8.0,  8.0]
NN = [41, 41, 41]
grid = FD3dGrid( NN, AA, BB )
atoms = Atoms( xyz_string=
  """
  1

  H  0.0  0.0  0.0
  """ )
V_Ps_loc = pot_Hps_HGH(atoms, grid)
Nstates = 1
Nelectrons = 1
Ham = Hamiltonian(atoms, grid, V_Ps_loc, Nelectrons=1)
\end{juliacode}

Result:
\begin{textcode}
... snipped
14      -0.2562610358   2.0600214368e-06   1.9590484379e-08
15      -0.2562602740   7.6184773828e-07   9.7155909123e-09
Convergence is achieved in 15 iterations
1      -0.0428298539
----------------------------
Total energy components
----------------------------
Kinetic =       0.2928408518
Ps_loc  =      -0.7625329979
Hartree =       0.2134318721
NN      =       0.0000000000
----------------------------
Total   =      -0.2562602740
\end{textcode}

TODO: try using denser grid. Study convergence of the total energy with respect
grid spacing.


\subsection{$\ce{H2}$ molecule}


Using the same grid as we have used for hydrogen atom.

\begin{juliacode}
atoms = Atoms( xyz_string=
  """
  2
  
  H   0.75  0.0  0.0
  H  -0.75  0.0  0.0
  """, in_bohr=true)
V_Ps_loc = pot_Hps_HGH(atoms, grid)
Nstates = 1
Nelectrons = 2
Ham = Hamiltonian(atoms, grid, V_Ps_loc, Nelectrons=Nelectrons)
\end{juliacode}

We need to calculate ion-ion interaction energy. Do this before SCF cycle.

\begin{juliacode}
Ham.energies.NN = calc_E_NN( [1.0, 1.0], atoms.positions )
\end{juliacode}

Result:
\begin{textcode}
... snipped
16      -0.5972202712   1.3097904592e-06   1.0969070600e-08
17      -0.5972197417   5.2951699026e-07   5.4868596190e-09
Convergence is achieved in 17 iterations
1      -0.1083633271
----------------------------
Total energy components
----------------------------
Kinetic =       0.8229371585
Ps_loc  =      -3.1339854624
Hartree =       1.0471618956
NN      =       0.6666666667
----------------------------
Total   =      -0.5972197417
\end{textcode}

TODO: try using denser grid. Study convergence of the total energy with respect
grid spacing.




\section{Kohn-Sham calculations}

Using XC

New Hamiltonian, include \jlinline{V_XC}.

Update the potential:

\begin{juliacode}
function update!( Ham::Hamiltonian, Rhoe::Vector{Float64} )
  Ham.rhoe = Rhoe
  Ham.V_Hartree = Poisson_solve_PCG( Ham.Laplacian, Ham.precLaplacian,
    -4*pi*Rhoe, 1000, verbose=false, TOL=1e-10 )
  Ham.V_XC = excVWN( Rhoe ) + Rhoe .* excpVWN( Rhoe )
  return
end
\end{juliacode}

Application of Hamiltonian

\begin{juliacode}
import Base: *
function *( Ham::Hamiltonian, psi::Matrix{Float64} )
  Nbasis = size(psi,1)
  Nstates = size(psi,2)
  Hpsi = zeros(Float64,Nbasis,Nstates)
  Hpsi = -0.5*Ham.Laplacian * psi
  for ist in 1:Nstates, ip in 1:Nbasis
    Hpsi[ip,ist] = Hpsi[ip,ist] + ( Ham.V_Ps_loc[ip] + Ham.V_Hartree[ip] +
                   Ham.V_XC[ip] ) * psi[ip,ist]
  end
  return Hpsi
end
\end{juliacode}

\section{Self-consistent loop}

The exchange correlation potential $V_{\mathrm{xc}}(\mathbf{r})$
is defined as functional derivative of
exchange correlation energy functional $E_{\mathrm{xc}}[\rho(\mathbf{r})]$
\begin{equation}
V_{\mathrm{xc}}(\mathbf{r}) = \frac{\delta E_{\mathrm{xc}}}{\delta \rho(\mathbf{r})}
\end{equation}
We do not have exact closed expression for these quantities, however several
approximations are available.