\chapter{Introduction to Julia programming language}

This chapter is intended to as an introduction to the Julia programming language.

This chapter assumes familiarity with command line interface.

\section{Installation}

Go to
{\scriptsize\url{https://julialang.org/downloads/}}
and download the suitable file for
your platform. For example, on 64 bit Linux OS, we can download the file
\txtinline{julia-1.x.x-linux-x86_64.tar.gz}
where \txtinline{1.x.x} referring to the version of Julia.
%
After you have downloaded the tarball you can unpack it.
%
\begin{bashcode}
tar xvf julia-1.x.x-linux-x86_64.tar.gz
\end{bashcode}
%
After unpacking the tarball, there should be a new folder called \txtinline{julia-1.x.x}.
You might want to put this directory under your home directory (or another directory of your
preference).

\section{Using Julia}

\subsection{Using Julia REPL}
Let's assume that you have put the Julia distribution under
your home directory. You can start the Julia interpreter by typing:
%
\begin{textcode}
/home/username/julia-1.x.x/bin/julia
\end{textcode}
%
You should see something like this in your terminal:
%
\begin{textcode}
$ julia 
               _
   _       _ _(_)_     |  Documentation: https://docs.julialang.org
  (_)     | (_) (_)    |
   _ _   _| |_  __ _   |  Type "?" for help, "]?" for Pkg help.
  | | | | | | |/ _` |  |
  | | |_| | | | (_| |  |  Version 1.1.1 (2019-05-16)
 _/ |\__'_|_|_|\__'_|  |  Official https://julialang.org/ release
|__/                   |

julia> 
\end{textcode}
%
This is called the Julia REPL (read-eval-print loop) or the Julia command prompt.
You can type the Julia program and see the output. This is useful for interactive
exploration or debugging the program.

The Julia code can be typed after the \txtinline{julia>} prompt. In this way,
we can write Julia code interactively.

Example Julia session
\begin{textcode}
julia> 1.2 + 3.4
4.6

julia> sin(2*pi)
-2.4492935982947064e-16

julia> sin(2*pi)^2 + cos(2*pi)^2
1.0
\end{textcode}

Using Unicode:
\begin{textcode}
julia> α = 1234;

julia> β = 3456;

julia> α * β
4264704
\end{textcode}

The symbol \jlinline{α} can be entered in the Julia console by typing \txtinline{\alpha} and
then followed by typing Tab key.
Another example is the \jlinline{∇} symbol can be entered by typing \txtinline{\nabla} followed
by typing Tab key.
Julia supports many Unicode symbols that can be used
as names or operators.
We can even use superscript symbol as in \jlinline{∇²}.
A list of supported Unicode symbols in Julia can be found at
{\footnotesize\url{https://docs.julialang.org/en/v1/manual/unicode-input/}}.

To exit the console we can type
\begin{textcode}
julia> exit()
\end{textcode}


\subsection{Julia script file}

We also can put the code in a text file with \txtinline{.jl} extension and
execute it with the command:
%
\begin{bashcode}
julia filename.jl
\end{bashcode}

For example, type the following code in a file named \txtinline{say_hello.jl}
\begin{juliacode}
function say_hello(name)
    println("Hello: ", name)
end
say_hello("efefer")
\end{juliacode}
The run it by using the following command in the usual terminal session
(not in a Julia console)
\begin{textcode}
julia say_hello.jl
\end{textcode}
and the following output will be displayed in the terminal:
\begin{textcode}
Hello: efefer
\end{textcode}




\section{Basic programming constructs}

Julia has similarities with several popular programming languages
such as Python, MATLAB, and R, to name a few.

\subsection{Displaying something}

Using \jlinline{println} and \jlinline{@printf}

\begin{juliacode}
print("Hey")
println("Hello")
\end{juliacode}

\begin{juliacode}
using Printf
@printf("Hey")
@printf("Hello\n")
\end{juliacode}
  
\subsection{Variables}

A variable, in Julia, is a name associated (or bound) to a value.
It's useful when you want to store a value (that you obtained after some math,
for example) for later use. For example:

\begin{juliacode}
# Assign the value 10 to the variable x
x = 10
  
# Doing math with x's value
x + 1
  
# Reassign x's value
x = 1 + 1
  
# You can assign values of other types, like strings of text
julia> x = "Hello World!"
\end{juliacode}

Julia provides an extremely flexible system for naming variables.
Variable names are case-sensitive, and have no semantic meaning
(that is, the language will not treat variables differently based on their names).

Variable names must begin with a letter (A-Z or a-z), underscore, or a subset of Unicode code points greater than 00A0; in particular, Unicode character categories Lu/Ll/Lt/Lm/Lo/Nl (letters), Sc/So (currency and other symbols), and a few other letter-like characters (e.g. a subset of the Sm math symbols) are allowed. Subsequent characters may also include ! and digits (0-9 and other characters in categories Nd/No), as well as other Unicode code points: diacritics and other modifying marks (categories Mn/Mc/Me/Sk), some punctuation connectors (category Pc), primes, and a few other characters.

The only explicitly disallowed names for variables are the names of built-in statements:

While Julia imposes few restrictions on valid names, it has become useful to adopt the following conventions:

Names of variables are in lower case.

Word separation can be indicated by underscores (\txtinline{'_'}), but use of underscores is discouraged unless the name would be hard to read otherwise.

Names of Types and Modules begin with a capital letter and word separation is shown with upper camel case instead of underscores.

Names of functions and macros are in lower case, without underscores.

Functions that write to their arguments have names that end in !. These are sometimes called "mutating" or "in-place" functions because they are intended to produce changes in their arguments after the function is called, not just return a value.


\subsection{Mathematical operators}

\begin{juliacode}
if a >= 1
  println("a is larger or equal to 1")
end
\end{juliacode}

Example code 3

Plotting:
\begin{juliacode}
using PGFPlotsX
using LaTeXStrings
include("init_FD1d_grid.jl")
function my_gaussian(x::Float64; α=1.0)
  return exp( -α*x^2 )
end
function main()
  A = -5.0
  B =  5.0
  Npoints = 8
  x, h = init_FD1d_grid( A, B, Npoints )

  NptsPlot = 200
  x_dense = range(A, stop=5, length=NptsPlot)

  f = @pgf(
    Axis( {height = "6cm", width = "10cm" },
      PlotInc( {mark="none"}, Coordinates(x_dense, my_gaussian.(x_dense)) ),
      LegendEntry(L"f(x)"),
      PlotInc( Coordinates(x, my_gaussian.(x)) ),
      LegendEntry(L"Sampled $f(x)$"),
    )
  )
  pgfsave("TEMP_gaussian_1d.pdf", f)
end
main()
\end{juliacode}
