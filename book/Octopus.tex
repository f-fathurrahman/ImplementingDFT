\chapter{Introduction to \textsf{Octopus} package}

\textsf{Octopus} is a software package that can be used to perform electronic
structure calculations based on various methods including using density functional theory and
time-dependent density functional theory. It also uses finite-difference discretization
of spatial domain and can do calculations in 1d, 2d and 3d systems with various
boundary condition. In this book, \textsf{Octopus} is used to provide results which can
be compared with our own calculation.

\section{Installation}

Octopus is a command line program and is distributed in the source form or source code.
The source code should be compiled before use. The binary distribution might be available in
some platform OS or Linux distros.

The following packages are the mandatory requirement for compiling \textsf{Octopus}.
\begin{itemize}
\item Autotools and GNU Make
\item C, C++ and Fortran compilers
\item Libxc
\end{itemize}

Download the source tarball, extract it, change working directory to the extracted
directory and do the following in the terminal:
\begin{bashcode}
autoreconf --install
./configure --prefix=path_to_install
make
make install
\end{bashcode}



\section{A short description of input file}

Octopus input is written in an input file named \txtinline{inp}. A typical
input file might have the following content. Note that comments (starting with character
\#) are allowed in the input file.

\begin{textcode}
# Type of calculation: gs means ground state calculation
CalculationMode = gs

# Dimension of the system
Dimensions = 3

# Number of periodic dimension
PeriodicDimensions = 0

# Electronic structure theory that is used
TheoryLevel = dft
# XC correlation function, here we specify the VWN exchange correlation
XCFunctional = lda_x + lda_c_vwn_1

# Box shape
BoxShape = parallelepiped

# Lsize is half of box size
# Here we have have a box of size 16x16x16 bohr
%Lsize
 8 | 8 | 8
%

# Grid spacing (in bohr, unless otherwise specified)
spacing = 0.4

# Some self-consistent field settings
MixField = density
MixingScheme = Linear
Mixing = 0.5

# Pseudopotential set
PseudopotentialSet = hgh_lda

# Do not alter potential
FilterPotentials = filter_none

# Atomic coordinates
% Coordinates
  "H" | -0.75 | 0.0 | 0.0
  "H" |  0.75 | 0.0 | 0.0
%
\end{textcode}

In this case we are having isolated or non-periodic system.
For full 3d periodic dimension we can write:
\begin{textcode}
PeriodicDimensions = 3
\end{textcode}

For solving Schroedinger equation only (single particle):
\begin{textcode}
TheoryLevel = independent_particles
\end{textcode}

For Hartree-only calculation
\begin{textcode}
XCFunctional = none
\end{textcode}

Note that by default the potentials will be filtered in order to minimize egg-box effect.
The following is the default setting:
\begin{textcode}
FilterPotentials = filter_TS
\end{textcode}
We will usually use unfiltered pseudopotential as we generally does not implement any
schemes to reduce eggbox effect in out Julia code.

To execute the \textsf{Octopus} program and redirect the standard output and error to a
file named \txtinline{LOG_calc} we can type:
\begin{textcode}
octopus >LOG_calc 2>&1
\end{textcode}
After successful executation, several files are produced. For our purposes the
\txtinline{LOG_calc} file that contains standard output is sufficient. Usually we
are interested in the converged total energy and eigenvalues which can be found
near the end of \txtinline{LOG_calc} file.

Redirect stdout to a file (\txtinline{>out}), and then redirect stderr to stdout
(\txtinline{2>&1}).

Poisson equation:
\begin{textcode}
PoissonSolver = cg_corrected
PoissonSolverMaxMultipole = 4
\end{textcode}
Default value of isolated system: \txtinline{isf} (interpolating scaling function).

Harmonic potential (user defined potential):
\begin{textcode}
# Harmonic potential, with 8 valence electrons or 4 states
% Species 
  'HO' | species_user_defined | potential_formula | "2*(x^2 + y^2 + z^2)" | valence | 8
%

# Define center of the potential
% Coordinates
  "HO" | 0 | 0 | 0
%
\end{textcode}


Gaussian potential, using variable r and pi.
\begin{textcode}
% Species 
  "X" | species_user_defined | potential_formula | "-exp(-r^2)/sqrt(1/pi)^3" | valence | 2
%

% Coordinates
  "X" | 0 | 0 | 0
%
\end{textcode}