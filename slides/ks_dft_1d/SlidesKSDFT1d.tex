\documentclass[10pt,aspectratio=169]{beamer}

%\documentclass[fleqn,aspectratio=169]{beamer}

\usepackage{amsmath, amssymb}
\usepackage{fancyvrb, color, graphicx, hyperref, url}

%\setbeamersize{text margin left=5pt, text margin right=5pt}

\setlength{\parskip}{\smallskipamount}
\setlength{\parindent}{0pt}

\usepackage{fontspec}
%\setmonofont{DejaVu Sans Mono}
\setmonofont{JuliaMono-Regular}

\usefonttheme[onlymath]{serif}

\usepackage{minted}
\newminted{python}{breaklines,fontsize=\footnotesize}
\newminted{julia}{breaklines,fontsize=\footnotesize}
\newminted{bash}{breaklines,fontsize=\footnotesize}
\newminted{text}{breaklines,fontsize=\footnotesize}

\newcommand{\txtinline}[1]{\mintinline[breaklines,fontsize=\footnotesize]{text}{#1}}
\newcommand{\pyinline}[1]{\mintinline[breaklines,fontsize=\footnotesize]{python}{#1}}
\newcommand{\jlinline}[1]{\mintinline[breaklines,fontsize=\footnotesize]{julia}{#1}}

\definecolor{mintedbg}{rgb}{0.95,0.95,0.95}
\usepackage{mdframed}

\BeforeBeginEnvironment{minted}{\begin{mdframed}[backgroundcolor=mintedbg,%
  rightline=false,leftline=false,topline=false,bottomline=false]}
\AfterEndEnvironment{minted}{\end{mdframed}}

% https://tex.stackexchange.com/questions/33969/changing-font-size-of-selected-slides-in-beamer

\usepackage{environ}
%
% Custom font for a frame.
%
\newcommand{\customframefont}[1]{
  \setbeamertemplate{itemize/enumerate body begin}{#1}
  \setbeamertemplate{itemize/enumerate subbody begin}{#1}
}

\NewEnviron{framefont}[1]{
  \customframefont{#1} % for itemize/enumerate
  {#1 % For the text outside itemize/enumerate
    \BODY
  }
  \customframefont{\normalsize}
}



\begin{document}

\title{Density Functional Theory Calculations for 1d Systems}
\author{Fadjar Fathurrahman}
\institute{
Teknik Fisika \\
Institut Teknologi Bandung
}
\date{}


\frame{\titlepage}


\begin{frame}
\frametitle{Introduction: Kohn-Sham equations (in 3d)}

\begin{equation*}
\mathcal{H}_{\mathrm{KS}} \psi_{i}(\mathbf{r}) = \epsilon_{i} \psi_{i}(\mathbf{r})
\end{equation*}

\begin{equation*}
\mathcal{H}_{\mathrm{KS}} =
-\frac{1}{2} \nabla^2 +
V_{\mathrm{ion}}(\mathbf{r}) +
V_{\mathrm{H}}(\mathbf{r}) + V_{\mathrm{xc}}(\mathbf{r})
\end{equation*}

\end{frame}


\begin{frame}
\frametitle{Electron density, Hartree and XC potentials}

\begin{equation*}
\rho(\mathbf{r}) = \sum_{i} f_{i} \psi_{i}(\mathbf{r}) \psi^{*}_{i}(\mathbf{r}) 
\end{equation*}

\begin{equation*}
V_{\mathrm{H}}(\mathbf{r}) = \int \frac{\rho(\mathbf{r}')}{\left| \mathbf{r} - \mathbf{r}' \right|}
\,\mathrm{d}\mathbf{r}'
\end{equation*}

\begin{equation*}
V_{\mathrm{xc}}(\mathbf{r}) = \frac{\delta E_{\mathrm{xc}}[\rho(\mathbf{r})]}%
{\delta \rho(\mathbf{r})}
\end{equation*}

\end{frame}



\begin{frame}
\frametitle{Self-consistency}

\begin{itemize}
\item Some terms of $\mathcal{H}_{\mathrm{KS}}$ depends on $\rho(\mathbf{r})$.
To calculate its matrix elements we need to know $\rho(\mathbf{r})$.
\item To calculate $\rho(\mathbf{r})$, we need to know $\psi_{i}(\mathbf{r})$
\item To calculate $\psi_{i}(\mathbf{r})$, we need to diagonalize $\mathcal{H}_{\mathrm{KS}}$
\end{itemize}

Kohn-Sham equations need to be solved self-consistently.

\end{frame}


% -----------------------------
\begin{frame}
\frametitle{Simplified, 1d Kohn-Sham model}

\begin{equation*}
E_{\mathrm{KS}}[\psi_{i}(x)] = E_{\mathrm{kin}} + E_{\mathrm{ion}} +
E_{\mathrm{H}} + E_{\mathrm{xc}}
\end{equation*}

\begin{equation*}
E_{\mathrm{kin}} = -\frac{1}{2} \int \psi^{*}_{i}(x)
\frac{\mathrm{d}^2}{\mathrm{d}x^2} \psi_{i}(x) \, \mathrm{d}x
\end{equation*}


\begin{equation*}
E_{\mathrm{ion}} = \int \psi^{*}(x) V_{\mathrm{ion}} \psi_{i}(x) \, \mathrm{d}x
\end{equation*}

\begin{equation*}
E_{\mathrm{H}} = \frac{1}{2} \int \int \frac{\rho(x) \rho(x')}{\left| x - x' \right|}
\,\mathrm{d}x \,\mathrm{d}x'
\end{equation*}

\begin{equation*}
E_{\mathrm{xc}} = \int \varepsilon_{\mathrm{xc}}[\rho(x)]
\rho(x) \, \mathrm{d}x
\end{equation*}

\end{frame}



% -------------------------
\begin{frame}
\frametitle{Kohn-Sham equations}

\begin{equation*}
H_{\mathrm{KS}} \psi_{i}(x) = \epsilon_{i} \psi_{i}(x)
\end{equation*}

\begin{equation*}
H_{\mathrm{KS}} =
-\frac{1}{2} \frac{\mathrm{d}^2}{\mathrm{d}x^2} +
V_{\mathrm{ion}}(x) + V_{\mathrm{H}}(x) + V_{\mathrm{xc}}(x)
\end{equation*}


\begin{equation*}
V_{\mathrm{H}}(x) = \int \frac{\rho(x')}{\left| x - x' \right|} \,\mathrm{d}x'
\end{equation*}


\begin{equation*}
V_{\mathrm{xc}}(x) = \frac{\delta E_{xc}[\rho(x)] }{\delta \rho(x)}
\end{equation*}

\end{frame}



\begin{frame}
\frametitle{Modification for Hartree potential}

To prevent singularity, we modify the Coulomb interaction:
\begin{equation*}
\frac{1}{\left| x - x' \right|} \rightarrow
\frac{1}{\sqrt{(x - x')^2 + a^2}}
\end{equation*}
where $a \neq 0$ is a parameter.

\end{frame}


\begin{frame}[fragile]
\frametitle{Real-space discretization}

We will make a grid of points in $x$ axis.
\begin{juliacode}
function init_FD1d_grid( x_min::Float64, x_max::Float64, N::Int64 )
    L = x_max - x_min
    h = L/(N-1)
    x = zeros(Float64,N)
    for i = 1:N
        x[i] = x_min + (i-1)*h
    end
    return x, h
end
init_FD1d_grid( X, N ) = init_FD1d_grid( X[1], X[2], N )
\end{juliacode}

\end{frame}


\begin{frame}[fragile]
\frametitle{Example discretization}

Suppose that we want to discretize $x$ domain from $x_{\mathrm{min}} = -8.0$
to $x_{\mathrm{max}} = -8.0$ with 9 points.
We call \pyinline{init_FD1d_grid} as:
\begin{juliacode}
xgrid, dx = init_FD1d_grid(-8.0, 8.0, 9)
\end{juliacode}

Result \jlinline{xgrid}:
\begin{textcode}
[-8.0, -6.0, -4.0, -2.0, 0.0, 2.0, 4.0, 6.0, 8.0]
\end{textcode}

Result \jlinline{dx}:
\begin{textcode}
2.0
\end{textcode}

\end{frame}


\begin{frame}[fragile]
\frametitle{Gaussian function}

\begin{equation*}
f(x) = e^{-\alpha x^2}
\end{equation*}

Take $\alpha = 1$.

Plot this function using discretized points on interval $[-5, 5]$.

Try various number of grid points.

\begin{juliacode}
function my_gaussian(x; α=1.0)
    return exp(-α*x^2)
end
\end{juliacode}

Note that we use default value of $\alpha = 1$. Try using different value of $\alpha$.

\end{frame}


% -----------------------------------------------
\begin{frame}[fragile]

\begin{juliacode}
A = -5.0; B =  5.0
xgrid, dx = init_FD1d_grid( A, B, N )
fx = my_gaussian.(xgrid)

Ndense = 200 # for reference
x_dense = range(A, stop=B, length=Ndense)
fx_dense = my_gaussian.(x_dense)

plt.clf()
plt.plot(xgrid, fx, marker="o", label=L"Sampled $f(x)$")
plt.plot(x_dense, fx_dense, label=L"f(x)")
plt.legend()
plt.grid()
plt.title(plot_title)
plt.savefig("IMG_gaussian_f_"*string(N)*".pdf")
\end{juliacode}

\end{frame}


% ------------------
\begin{frame}

{\centering
\includegraphics[width=0.8\textwidth]{../../codes/ks_dft_1d/IMG_gaussian_f_15.pdf}
\par}

\end{frame}


\begin{frame}

{\centering
\includegraphics[width=0.8\textwidth]{../../codes/ks_dft_1d/IMG_gaussian_f_51.pdf}
\par}

\end{frame}


\begin{frame}[fragile]
\frametitle{Approximating integral}

\begin{equation*}
I = \int e^{-\alpha x^2}\, \mathrm{d}x = \sqrt{ \frac{\pi}{\alpha} }
\end{equation*}
\begin{equation*}
I \approx \sum_{i} e^{-\alpha x_{i}^2} \Delta x
\end{equation*}

\begin{juliacode}
function do_integrate(N::Int64)
    A = -5.0
    B = 5.0
    α = 1.0
    xgrid, dx = init_FD1d_grid( A, B, N )
    fx = my_gaussian.(xgrid, α=α)
    exact_res = sqrt(π)/sqrt(α)
    num_res = sum(fx)*dx # approximate the integral
    @printf("%5d %18.10f %10.5e\n", N, num_res, abs(exact_res - num_res))
end

for N in [10, 20, 30, 40, 50]
    do_integrate(N)
end
\end{juliacode}

\end{frame}



% ---------------------------------------------
\begin{frame}
\frametitle{Finite difference approximation}


\end{frame}




\end{document}
