\chapter{Introduction to Julia programming language}


\section{Basic syntax}


\section{Mathematical operators}

\begin{juliacode}
if a >= 1
  println("a is larger or equal to 1")
end
\end{juliacode}


\section{Test Unicode symbol}


Example code 1
\begin{juliacode}
∇2 = kron(D2x, speye(Ny)) + kron(speye(Nx), D2y)
\end{juliacode}

Example code 2
\begin{juliacode}
∇2 = kron(D2x, speye(Ny)) + kron(speye(Nx), D2y)
∇2 = D2x⊗IIy⊗IIz + IIx⊗D2y⊗IIz + IIx⊗IIy⊗D2z 
\end{juliacode}

Example code 3
\begin{juliacode}
using PGFPlotsX
using LaTeXStrings
include("init_FD1d_grid.jl")
function my_gaussian(x::Float64; α=1.0)
  return exp( -α*x^2 )
end
function main()
  A = -5.0
  B =  5.0
  Npoints = 8
  x, h = init_FD1d_grid( A, B, Npoints )

  NptsPlot = 200
  x_dense = range(A, stop=5, length=NptsPlot)

  f = @pgf(
    Axis( {height = "6cm", width = "10cm" },
      PlotInc( {mark="none"}, Coordinates(x_dense, my_gaussian.(x_dense)) ),
      LegendEntry(L"f(x)"),
      PlotInc( Coordinates(x, my_gaussian.(x)) ),
      LegendEntry(L"Sampled $f(x)$"),
    )
  )
  pgfsave("TEMP_gaussian_1d.pdf", f)
end
main()
\end{juliacode}
