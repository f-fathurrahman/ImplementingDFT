\chapter{Introduction to Julia programming language}

This chapter is intended to as an introduction to the Julia programming language.

This chapter assumes familiarity with command line interface.

\section{Installation}

Go to
{\scriptsize\url{https://julialang.org/downloads/}}
and download the suitable file for
your platform. For example, on 64 bit Linux OS, we can download the file
\txtinline{julia-1.x.x-linux-x86_64.tar.gz}
where \txtinline{1.x.x} referring to the version of Julia.
%
After you have downloaded the tarball you can unpack it.
%
\begin{bashcode}
tar xvf julia-1.x.x-linux-x86_64.tar.gz
\end{bashcode}
%
After unpacking the tarball, there should be a new folder called \txtinline{julia-1.x.x}.
You might want to put this directory under your home directory (or another directory of your
preference).

\section{Using Julia}

\subsection{Using Julia REPL}
Let's assume that you have put the Julia distribution under
your home directory. You can start the Julia interpreter by typing:
%
\begin{textcode}
/home/username/julia-1.x.x/bin/julia
\end{textcode}
%
You should see something like this in your terminal:
%
\begin{textcode}
$ julia 
               _
   _       _ _(_)_     |  Documentation: https://docs.julialang.org
  (_)     | (_) (_)    |
   _ _   _| |_  __ _   |  Type "?" for help, "]?" for Pkg help.
  | | | | | | |/ _` |  |
  | | |_| | | | (_| |  |  Version 1.1.1 (2019-05-16)
 _/ |\__'_|_|_|\__'_|  |  Official https://julialang.org/ release
|__/                   |

julia> 
\end{textcode}
%
This is called the Julia REPL (read-eval-print loop) or the Julia command prompt.
You can type the Julia program and see the output. This is useful for interactive
exploration or debugging the program.

The Julia code can be typed after the \txtinline{julia>} prompt. In this way,
we can write Julia code interactively.


Example Julia session
\begin{textcode}
julia> 1.2 + 3.4
4.6

julia> sin(2*pi)
-2.4492935982947064e-16

julia> sin(2*pi)^2 + cos(2*pi)^2
1.0
\end{textcode}

Using Unicode:
\begin{textcode}
julia> α = 1234;

julia> β = 3456;

julia> α * β
4264704
\end{textcode}

To exit type
\begin{textcode}
julia> exit()
\end{textcode}


\subsection{Julia script file}

In a text file with \txtinline{.jl} extension.

You can experiment with Julia REPL by typing \txtinline{julia}
at terminal:

We also can put the code in a text file with \txtinline{.jl} extension and
execute it with the command:
%
\begin{bashcode}
julia filename.jl
\end{bashcode}

The following code
\begin{juliacode}
function say_hello(name)
    println("Hello: ", name)
end
say_hello("efefer")
\end{juliacode}


\section{Basic programming construct}

Julia has similarities with several popular programming languages
such as Julia, MATLAB, and R, to name a few.


\section{Mathematical operators}

\begin{juliacode}
if a >= 1
  println("a is larger or equal to 1")
end
\end{juliacode}

Example code 3

\begin{juliacode}
using PGFPlotsX
using LaTeXStrings
include("init_FD1d_grid.jl")
function my_gaussian(x::Float64; α=1.0)
  return exp( -α*x^2 )
end
function main()
  A = -5.0
  B =  5.0
  Npoints = 8
  x, h = init_FD1d_grid( A, B, Npoints )

  NptsPlot = 200
  x_dense = range(A, stop=5, length=NptsPlot)

  f = @pgf(
    Axis( {height = "6cm", width = "10cm" },
      PlotInc( {mark="none"}, Coordinates(x_dense, my_gaussian.(x_dense)) ),
      LegendEntry(L"f(x)"),
      PlotInc( Coordinates(x, my_gaussian.(x)) ),
      LegendEntry(L"Sampled $f(x)$"),
    )
  )
  pgfsave("TEMP_gaussian_1d.pdf", f)
end
main()
\end{juliacode}
